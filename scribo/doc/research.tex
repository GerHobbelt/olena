%% Copyright (C) 2011 EPITA Research and Development Laboratory (LRDE)
%%
%% This file is part of Olena.
%%
%% Olena is free software: you can redistribute it and/or modify it under
%% the terms of the GNU General Public License as published by the Free
%% Software Foundation, version 2 of the License.
%%
%% Olena is distributed in the hope that it will be useful,
%% but WITHOUT ANY WARRANTY; without even the implied warranty of
%% MERCHANTABILITY or FITNESS FOR A PARTICULAR PURPOSE.  See the GNU
%% General Public License for more details.
%%
%% You should have received a copy of the GNU General Public License
%% along with Olena.  If not, see <http://www.gnu.org/licenses/>.

\documentclass[a4]{book}

%\usepackage{hevea}

\usepackage{html}
\usepackage{hyperref}
\usepackage{graphicx}
\usepackage{makeidx}
\usepackage{xcolor}
\usepackage{color}

\title{SCRIBO\\
  \large{Research report} }
\author{LRDE}
\date{}
\makeindex


\begin{document}

\maketitle



%===========================================
%===========================================
%===========================================
\chapter{Preprocessing}



%*******************************************
%*******************************************
\section{Show-through removal}


%*******************************************
%*******************************************
\section{Color to grayscale conversion}

2 formulas tested :
\begin{itemize}
\item $R + G + V$
\item $0.299 * R + 0.587 * G + 0.114 * B$
\end{itemize}


%*******************************************
%*******************************************
\section{Binarization}



%...........................................
\subsection{Sauvola}
\par{Sauvola}

\cite{Sauvola}

Best published method for documents.

Parameters set up according to \cite{Badekas}.

\par{Sauvola Multi-scale}

Implemented with integral images. \cite{Faisal.integral_images}

\par{Sauvola 3-channels}



%*******************************************
%*******************************************
\section{Background/Foreground identification}



%*******************************************
%*******************************************
\section{Unskew}



%*******************************************
%*******************************************
\section{Denoising}



%*******************************************
%*******************************************
\section{Delimitors}

%...........................................
\subsection{Lines}

%...........................................
\subsection{Tab-stops and whitespaces}

File concerned : scribo/primitive/extract/separators\_non\_visible.hh

First attempt to retrieve tab-stops/whitespaces delimitors.  In order
to limit false positive, the components are dilated horizontaly prior
the algorithm.

False positive were still too numerous in the core paragraphes.


File concerned : scribo/primitive/extract/alignments.hh

In order to avoid too much false positive, the text is grouped once
(almost by word).  To limit connections between paragraphs, the rules
used to connect components is as follows : lookup for the closest left
neighbor until a maximum distance compute with the formula (w / 2.0f)
+ (dmax_factor_ * h), where w and h are respectively the width and the
height of the component. dmax_factor_ is a user defined parameter set
to 1. Functor primitive::link::internal::dmax_default is used and
implement that rule..

We tried to find tabstops and whitespaces without grouping first but
there were too much false positive inside paragraphs.  Grouping may be
a problem some times since if two paragraphs are too close to
eachother, they may already connect...


%===========================================
%===========================================
%===========================================
\chapter{Text extraction}

%*******************************************
%*******************************************
\section{lines}

%...........................................
\subsection{Component labeling}

%...........................................
\subsection{Component grouping}

%...........................................
\subsection{Line reconstruction}



%*******************************************
%*******************************************
\section{paragraphs/text blocks}


%===========================================
%===========================================
%===========================================
\chapter{Non-text object extraction}

%*******************************************
%*******************************************
\section{Background learning}


%===========================================
%===========================================
%===========================================
\chapter{Text recognition (OCR)}

%*******************************************
%*******************************************
\section{Tesseract Integration}


%*******************************************
%*******************************************
\section{Text cleanup}


%===========================================
%===========================================
%===========================================
\chapter{Data structures}

%*******************************************
%*******************************************
\section{Component\_set}
\subsection{Component\_info}

%*******************************************
%*******************************************
\section{object\_links}

%*******************************************
%*******************************************
\section{object\_groups}



%*******************************************
%*******************************************
\section{line\_set}

%...........................................
\subsection{line\_info}




%*******************************************
%*******************************************
\section{paragraph\_set}

%...........................................
\subsection{paragraph\_info}

\end{document}

