\documentclass{report}
\usepackage{graphicx}
\usepackage{listings}
\usepackage{makeidx}
\usepackage{xcolor}
\usepackage{color}

\newcommand{\img}[4]{
\begin{figure}[ht!]
	\begin{center}
      \includegraphics[width=#2]{figures/#1}
     \caption{#4\label{fig:#1}}
   \end{center}
 \end{figure}
}

\title{Olena - Tutorial
}
\author{LRDE}
\date{}

%%%LISTINGS SETTINGS
\lstset{frameround=fttt}
\lstloadlanguages{C++}
\lstset{language=C++}
\lstset{backgroundcolor=\color{black},
basicstyle=\bfseries\color{white},
identifierstyle=\color{green},
stringstyle=\color{yellow},
commentstyle=\color{red},
showstringspaces=false,linewidth=14cm}


\begin{document}

\tableofcontents

ajouter dans => milena/doc/tutorial  |
-------------------------------------------------

\chapter{Foreword}

The following tutorial talks about 2D images. We chose that kind of images
because it is easier to understand and this is the
most used. \\

FIXME
[dessin de grille 2d, colonnes/lignes numerotees + repere x/y]
Intersection <=> point 2d <=> milieu d'un pixel \\

Since examples are based on 2D images pixels are actually "points" however we
will call them "sites" which is the most generic name.\\

Here is also a list of common variable name conventions:
\begin{figure}[ht!]
  \begin{tabular}{|l|l|}
  \hline
  \textbf{Object} & \textbf{Variable name} \\ \hline
  Site & p                            \\ \hline
  Value & v                           \\ \hline
  Neighboor & n                       \\ \hline
  A site close to another site p & q  \\ \hline
  \end{tabular}
\end{figure}

Methods provided by the objects in the library are in constant time. If you need
a specific method but you cannot find it, you may find an algorithm which can
compute the information you need.

Olena is organized in a namespace hierarchy.  Everything is declared by Olena
within the 'oln::' namespace, and possibly a sub-namespace such as
'oln::arith::' (arithmetic operations on images), 'oln::morpho::' (morphological
operations), etc.  For the sake of simplicity, we will neglect the 'oln::'
prefix in all the code examples.


\chapter{Concepts}
//FIXME lister \underline{TOUS} les concepts (core/concepts)

\section{Site set}

Site sets are used:
\begin{enumerate}
  \item To define an image definition domain.
  \item As Site container.
\end{enumerate}

Here is a list of all the site set concepts which can be found in
core/site\_set:

\begin{tabular}{|l|p{8cm}|}
\hline
Site set & Description \\ \hline

p\_key & \\ \hline
p\_priority & \\ \hline
p\_box & \\ \hline
p\_bgraph & \\ \hline
p\_double & \\ \hline
p\_if & \\ \hline
p\_line\_graph & \\ \hline
p\_queue\_fast & \\ \hline
p\_set & \\ \hline
p\_faces & \\ \hline
p\_queue & \\ \hline
p\_set\_of & \\ \hline
line2d & \\ \hline
p\_complex & \\ \hline
p\_graph & \\ \hline
p\_image & \\ \hline
p\_mutable\_array\_of & \\ \hline
p\_run & \\ \hline
p\_vaccess & \\ \hline
\end{tabular}

\subsection{Basic interface}
Common basic interface:\\

\begin{tabular}{|l|l|l|l|p{4cm}|}
\hline
Return Type & Name & Arguments & Const & Commor type automatically from the
given container type
passed as parameter. These macros can be used with any container like images or
site sets.ents \\ \hline

bool & is\_valid & - & X & Returns true if it has been initialized. The
default constructor does not do it. \\ \hline

bool & has & const P\& p & X &  \\ \hline
\end{tabular} \\


\subsection{Optional interface}
Site sets may have other methods depending on their type: \\

\begin{tabular}{|l|l|l|l|p{4cm}|}
\hline
Return Type & Name & Arguments & Const & Comments \\ \hline

size\_t & nsites & - & - & \\ \hline
const Box\& & bbox & - & X &  Bounding box. Available only on grid site sets.
\\ \hline
\end{tabular} \\

The previous methods are available depending on the site set. A box
will have the bbox() method since it can be retrived in constant time: a box
is it's own bounding box (see \ref{fig:box_bbox}). A p\_array does not have this
method since sites do not have to be adjacent. Maintaining such information, in
order to keep getting the bbox in constant time, would be time and memory
consuming. Instead of providing a method directly in p\_array, an algorithm is
available if this information needed (see \ref{fig:parray_bbox}).
P\_array and box both have a nsites method since the internal structure allow a
constant time retrieval.

\subsubsection*{Sample code}

\begin{figure}[ht!]
  \begin{lstlisting}[frame=single]
  p_array<point2d> arr;

  // The bbox is computed thanks to bbox() algorithm.
  box2d box = geom::bbox(arr);
  std::cout << box << std::endl;

  // p_array provides nsites(),
  // it can be retrieved in constant time.
  std::cout << "nsites = " << arr.nsites() << std::endl;
  \end{lstlisting}
  \caption{How to retrieve information from a p\_array.\label{fig:parray_bbox}}
\end{figure}

\begin{figure}[ht!]
  \begin{lstlisting}[frame=single]
  box2d b(2,3);

  // The bbox can be retrived in constant time.
  std::cout << b.bbox() << std::endl;

  // nsites can be retrieved in constant time.
  std::cout << "nsites = " << b.nsites() << std::endl;
  \end{lstlisting}
  \caption{How to retrieve information from a box.\label{fig:box_bbox}}
\end{figure}

\clearpage
\newpage
\section{Image}

An image is composed both of:
\begin{itemize}
\item A function $$
f : \left\{
  \begin{array}{lll}
    Site &\rightarrow & Value \\
    p & \mapsto & v
  \end{array}
\right.
$$
\item A site set, also called the "domain".
\end{itemize}

Every image type is defined on a specific site type. An image2d will always
have a domain defined by a box2d.
The Value set, which includes all the possible values a site can have, is also
called "destination" set.

An image has a virtual border which is defined thanks to its domain. The
 border is virtual since the image can have an extended domain as well.
That one is optional, it defines sites outside the virtual border which is
useful in algorithms when working with sites being part of the domain but close
to the borders. The virtual border can be defined thanks to a function, an
image or a site set.

//FIXME: remove this line
ici, site <=> Point2d


[sample code]

In order to create a 2D image, you have two possibilites:
\begin{lstlisting}[frame=single]
  // which builds an empty image;
  image2d<int> img1a;
  // which builds an image with 6 sites
  image2d<int> img1b(box2d(2, 3));
\end{lstlisting}

The empty image has no data and its definition domain is still unset.  We do
not know yet the number of sites it contains. However, it is really useful to
have such an "empty image" because it is a placeholder for the result of some
processing, or another image.

Trying to access the site value from an empty image leads to an error at
run-time.


\begin{lstlisting}[frame=single]
  box2d b(2,3);
  image2d<int> ima(b); // Define the domain of the image.

  cout << b << std::endl; // Display b

  cout << ima.domain() << std::endl; // Display b too
\end{lstlisting}

\begin{lstlisting}[frame=single]
  box2d b(2,3);
  image2d<int> ima(b);
  point2d p(1, 2);

  ima.at(1,2) = 9; // The value is returned by reference
                   // and can be changed.
  cout << ima(p) << std::endl; // prints 9

  ima(p) = 2; // The value is returned by reference
              // and can be changed.
  cout << ima(p) << std::endl; // prints 2
\end{lstlisting}


To know if a point belongs to an image domain or not, we can run this short
test:
\begin{lstlisting}[frame=single]
point2d p(9, 9);

// which gives 'true'.
std::cout << (imga.has(p) ? "true" : "false") << std::endl;
\end{lstlisting}

Since the notion of point is independent from the image it applies on, we can
form expressions where p is used on several images:
\begin{lstlisting}[frame=single]
// At index (9, 9), both values change.
imga(p) = 'M', imgb(p) = 'W';

debug::println(imga);
debug::println(imgb);
\end{lstlisting}


Images do not actually store the data in the class. This is a pointer
to an allocated space which can be shared with other objects. Once an image is
assigned to another one, the two images share the same data so they have the
same ID and point to the same memory space.
Therefore, assigning an image to another one is NOT a costly operation. The new
variable behaves like some mathematical variable.  Put differently it is just a
name to designate an image:
\begin{lstlisting}[frame=single]
  image2d<int> ima1(box2d(2, 3));
  image2d<int> ima2;
  point2d p(1,2);

  ima2 = ima1; // ima1.id() == ima2.id()
               // and both point to the same memory area.

  ima2(p) = 2; // ima1 is modified as well.

  // prints "2 - 2"
  std::cout << ima2(p) << " - " << ima1(p) << std::endl;
  // prints "true"
  std::cout << (ima2.id() == ima1.id()) << std::endl;
\end{lstlisting}

If a deep copy of the image is needed, a method clone() is available:
\begin{lstlisting}[frame=single]
  image2d<int> ima3 = ima1.clone(); // Makes a deep copy.

  ima3(p) = 3;

  // prints "3 - 2"
  std::cout << ima3(p) << " - " << ima1(p) << std::endl;
  // prints "false"
  std::cout << (ima3.id() == ima1.id()) << std::endl;
\end{lstlisting}

[Illustration : grille + intersection + pmin() + pmax() + distance entre 2
points en x et en y = 1]\\

In the Olena library, all image types behave like image2d:
\begin{itemize}
\item An "empty" image actually is a mathematical variable.

      $\rightarrow$ just think in a mathemetical way when dealing with images;

\item No dynamic memory allocation/deallocation is required.
    the user never has to use "new / delete" (the C++ equivalent for the C
    "malloc / free") so she does not have to manipulate pointers or to directly
    access memory.
    
    $\rightarrow$ Olena prevents the user from making mistakes;

\item Image data/values can be shared between several variables and the memory
    used for image data is handled by the library.
    
    $\rightarrow$ Memory management is automatic.
\end{itemize}


\subsection{Interface}

\begin{tabular}{|l|l|l|l|p{4cm}|}
\hline
Return Type & Name & Arguments & Const & Comments \\ \hline

site\_set & domain & - & X - &  \\ \hline
const Value\& & operator() & const point\& p & X & Used for reading. \\ \hline
Value\& & operator() & const point\& p & - & Used for writing. \\ \hline
const P\& & at & unsigned x,
         unsigned y & X & Used for reading. \\ \hline
P\& & at & unsigned x,
         unsigned y & - & Used for writing. \\ \hline
bool & has & const Point\& p & X & \\ \hline
bool & has\_data & - & X & Returns true if the domain is defined. \\ \hline
site\_id & id & - & X & Return the Id of the underlying shared data. \\ \hline
FIXME & destination & - & X & Value set of all the possible site values in this
Image. \\ \hline
site\_set & bbox & - & - & Returns the bounding box of the domain. \\ \hline
site\_set & bbox\_large & - & - & Returns the bouding box of the domain and the
extended domain. \\ \hline

\end{tabular}


\newpage
\section{Neighborhood}



\newpage
\section{Window}



\chapter{Iterators}

Every objects 
Each container object in Olena like site sets or images have iterators.
There are usually three kinds:
\begin{itemize}
\item \textbf{fwd\_iter}, depends on the container,
\item \textbf{bkd\_iter}, iterates like forward but to the opposite way,
\item \textbf{iter}, usually the same as fwd\_iter. It is guaranteed to
iterate all over the elements.
\end{itemize}

The iterator type name depends on the data pointed by it: \\

\begin{tabular}{|l|l|l|l|p{4cm}|}
\hline
Data type & Iterator Names \\ \hline
Site & fwd\_piter, bkd\_piter, piter \\ \hline
Value & fwd\_viter, bkd\_viter, viter \\ \hline
\end{tabular} \\

As you may have noticed, according to the data type, the word "iter" is prefixed
by the usual name variable used for that data type. Sites variables are usually
called 'p' so the proper iterator is "piter".\\


An iterator has the following interface: \\

\begin{tabular}{|l|l|l|l|p{4cm}|}
\hline
Return Type & Name & Arguments & Const & Comments \\ \hline

void & start & - & - & \\ \hline
void & next & - & - & \\ \hline
bool & is\_valid & - & - & Return false if created with the default
constructor and not associated to a proper container.\\ \hline
\end{tabular} \\


Example of different forward iterations:
\begin{itemize}
  \item box2d: from top to bottom then from left to right.
  \item p\_array<point2d>: from left to right.
\end{itemize}

A for\_all() macro is available to iterate over all the sites
(Fig. \ref{fig:for_all}). \\


\begin{figure}[ht!]
  \begin{lstlisting}[frame=single]
  box2d b(3, 2);
  mln_piter(box2d) p(b);

  for_all(p)
    std::cout << p; //prints every site coordinates.
  \end{lstlisting}
  \caption{Use of the for\_all() macro.\label{fig:for_all}}
\end{figure}

Note that when you declare an iterator, prefer using the "mln\_*iter" macros.
They resolve the iterator type automatically from the given container type
passed as parameter. These macros can be used with any container like images or
site sets.
(\ref{fig:iter_allcontainers}).

\begin{figure}[ht!]
  \begin{lstlisting}[frame=single]
  image2d<int> ima(box2d(2, 3));

  mln_piter(box2d) p(ima.domain());
  for_all(p)
    std::cout << p << std::endl;

  mln_viter(image2d<int>) v(ima.destination());
  for_all(v)
    std::cout << v << std::endl;
  \end{lstlisting}
  \caption{mln\_*iter macros can be used with any
containers.\label{fig:iter_allcontainers}}
\end{figure} 


\chapter{Basic operations}
//FIXME : illustrer
\begin{itemize}
  \item level::clone(), creates a deep copy of an object. Any shared data is
duplicated.
  \item level::fill(), fill an object with a value (fig. \ref{fig:fill_impl}).
  \item level::paste(), paste object data to another object (fig.
\ref{fig:paste_impl})

  \item labeling::blobs(), find and label the different components of an image.
\end{itemize}

First, create an image:
\begin{lstlisting}[frame=single]
  image2d<char> imga(0, 0, 20, 20);
\end{lstlisting}

Memory has been allocated so data can be stored but site values
have not been initialized yet.  So we fill img with the value 'a':
\begin{lstlisting}[frame=single]
  level::fill(inplace(imga), 'a');
\end{lstlisting}

The "fill" algorithm is located in the sub-namespace "level" since this
algorithm deals with the "level" of site values.

Note that the term "level" refers to the fact that an image can be considered as
a landscape where the elevation at a particular location/site is given by
the corresponding site value.

The full name of this routine is "oln::level::fill".  To access to a particular
algorithm, the proper file shall be included. The file names of algorithms
strictly map their C++ name; so oln::level::fill is defined in the file
"oln/level/fill.hh".

Most algorithms in Olena are constructed following the classical scheme: "output
algo(input)", where the input image is only read. However some few algorithms
take an input image in order to modify it.  To enforce this particular feature,
the user shall explicitly state that the image is provided so that its data is
modified "inplace". The algorithm call shall be "level::fill(inplace(ima),
val)". When forgetting the "inplace(..)" statement it does not compile.

We then define below a second image to play with.  As you can see this image has
data for the sites (5, 5) to (14, 14) (so it has 100 sites).  The definition
domain of a 2D image can start from any sites, even a negative one.

\begin{lstlisting}[frame=single]
  image1d<char> imgb(5, 5, 14, 14);

  // We initialize the image values.
  level::fill(inplace(imgb), 'b');

  // Last we now paste the contents of img3b in img3a...
  level::paste(imgb, inplace(imga));

  // ...and print the result.
  debug::println(imga);
\end{lstlisting}

Before pasting, the couple of images looked like:

//FIXME : ajouter des zolies zimages.

so after pasting we get:

//FIXME : ajouter des zolies zimages again.

With this simple example we can see that images defined on different domains (or
set of sites) can interoperate.  The set of sites of an image is defined and
can be accessed and printed. The following code:

\begin{lstlisting}[frame=single]
  std::cout << "imga.domain() = " << imga.domain()
            << std::endl;
  std::cout << "imgb.domain() = " << imgb.domain()
            << std::endl;
\end{lstlisting}

Gives:
\begin{lstlisting}[frame=single]
  imga.domain() = { (0,0) .. (19,19) }
  imgb.domain() = { (5,5) .. (14,14) }
\end{lstlisting}

The notion of site sets plays an important role in Olena. Many tests are
performed at run-time to ensure that the program is correct.

For instance, the algorithm level::paste tests that the set of sites of imgb
(whose values are to be pasted) is a subset of the destination image.


\begin{figure}[ht!]
  \begin{lstlisting}[frame=single]
template <typename I>
void fill(I& ima, mln_value(I) v)
{
  mln_piter(I) p(ima.domain());
  for_all(p)
    ima(p) = v;
}
  \end{lstlisting}
  \caption{Implementation of the fill routine.\label{fig:fill_impl}}
\end{figure} 


\begin{figure}[ht!]
  \begin{lstlisting}[frame=single]
template <typename I, typename J>
void paste(const I& data, J& dest)
{
  mln_piter(I) p(data.domain());
  for_all(p)
    dest(p) = data(p);
}
  \end{lstlisting}
  \caption{Implementation of the paste routine.\label{fig:paste_impl}}
\end{figure}


\section{Working with parts of an image}

Sometime it may be interesting to work only on some part of the image or to
extract only a sub set of that image. Olena enables that thoughout out the
operator '$|$'.

Three kinds of that operator exist:\\

\begin{tabular}{|l|l|l|l|p{4cm}|}
\hline
Prototype & Comments \\ \hline

Image $|$ Sub Domain & Create a new image.\\ \hline
Image $|$ Function\_p2b & Do not create a new image but create a morpher.\\
\hline
Function\_p2v $|$ Sub Domain & Do not create a new image but create a morpher.\\
\hline
\end{tabular} \\

A Sub Domain can be a site set, an image or any value returned by this
operator.
For a given site, Function\_p2v returns a Value and Function\_p2b returns a
boolean. These functions. are actually a sort of predicate. You can easily get
of function of function\_p2v kind thanks to pw::value(Image). It returns the
point to value function used in the given image. C functions can also be used as
predicate by passing the function pointer.

You can easily get a function\_p2b by comparing the value returned 
by a function\_p2v to another Value.
The following sample code illustrates this feature.

In this section, all along the examples, the image "ima" will refer to the
following declaration:
\begin{lstlisting}[frame=single]
  bool vals[6][5] = {
    {0, 1, 1, 0, 0},
    {0, 1, 1, 0, 0},
    {0, 0, 0, 0, 0},
    {1, 1, 0, 1, 0},
    {1, 0, 1, 1, 1},
    {1, 0, 0, 0, 0}
  };
  image2d<bool> ima = make::image2d(vals);
\end{lstlisting}

A simple example is to fill only a part of an image with a specific value:
\begin{lstlisting}[frame=single]
p_array2d<bool> arr;

// We add two points in the array.
arr.append(point2d(0, 1));
arr.append(point2d(4, 0));

// We restrict the image to the sites
// contained in arr and fill these ones
// with 0.
// We must call "inplace" here.
fill(inplace(ima | arr), 0);


mln_VAR(ima2, ima | arr);
// We do not need to call "inplace" here.
fill(ima2, 0);
\end{lstlisting}

The two next examples extract a specific component from an image and fill a new
image with red only in the extracted component's domain.
\begin{lstlisting}[frame=single]
  using namespace mln;
  using value::int_u8;
  using value::rgb8;

  // Find and label the different components.
  int_u8 nlabels;
  image2d<int_u8> lab = labeling::blobs(ima, c4(), nlabels);

  // Store a boolean image. True if the site is part of
  // component 2, false otherwise.
  mln_VAR(lab_2b, lab | (pw::value(lab) == 2));

  // Get the sub site set containing only the sites
  // part of component 2.
  mln_VAR(lab_2, lab_2b.domain(1));

  // Fill the sites of component 2 with red.
  fill(lab_2, color::red);
\end{lstlisting}

The previous example can be written more quickly:
\begin{lstlisting}[frame=single]
  using namespace mln;
  using value::int_u8;
  using value::rgb8;

  // Find and label the different components.
  int_u8 nlabels;
  image2d<int_u8> lab = labeling::blobs(ima, c4(), nlabels);

  // Fill the sites of component 2 with red.
  fill(inplace(lab.domain(2)), color::red);
\end{lstlisting}


\chapter{Graphes and images}



\end{document}