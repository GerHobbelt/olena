% $Header: /cvsroot/latex-beamer/latex-beamer/examples/beamerexample1.tex,v 1.46 2004/10/07 20:53:07 tantau Exp $

\documentclass{beamer}
%\usepackage{beamerthemesplit}

%\documentclass{article}
%\usepackage[envcountsect]{beamerarticle}

% Do NOT take this file as a template for your own talks. Use a file
% in the directory solutions instead. They are much better suited.

% Try the class options [notes], [notes=only], [trans], [handout],
% [red], [compress], [draft] and see what happens!

% Copyright 2003 by Till Tantau <tantau@users.sourceforge.net>.
%
% This program can be redistributed and/or modified under the terms
% of the LaTeX Project Public License Distributed from CTAN
% archives in directory macros/latex/base/lppl.txt.

% For a green structure color use:
%\colorlet{structure}{green!50!black}

 \mode<article> % only for the article version
 {
   \usepackage{fullpage}
   \usepackage{myhyperref}
 }


 \mode<presentation>
 {
   \setbeamertemplate{background canvas}[vertical shading][bottom=red!10,top=blue!10]
   \usetheme{Warsaw}
   \usefonttheme[onlysmall]{structurebold}


% Put the page number too.
\defbeamertemplate*{footline}{infolines theme without institution}
{
  \leavevmode%
  \hbox{%
  \begin{beamercolorbox}[wd=.333333\paperwidth,ht=2.25ex,dp=1.125ex,center]{author in head/foot}%
    \usebeamerfont{author in head/foot}\insertshortauthor
  \end{beamercolorbox}%
  \begin{beamercolorbox}[wd=.333333\paperwidth,ht=2.25ex,dp=1.125ex,center]{title in head/foot}%
    \usebeamerfont{title in head/foot}\insertshorttitle
  \end{beamercolorbox}%
  \begin{beamercolorbox}[wd=.333333\paperwidth,ht=2.25ex,dp=1.125ex,right]{date 
in head/foot}%
    \usebeamerfont{date in head/foot}\insertshortdate{}\hspace*{2em}
    \insertframenumber{} / \inserttotalframenumber\hspace*{2ex}
  \end{beamercolorbox}}%
  \vskip0pt%
}


}
% end of mode presentation



%\setbeamercolor{math text}{fg=green!50!black}
%\setbeamercolor{normal text in math text}{parent=math text}

\usepackage{pgf,pgfarrows,pgfnodes,pgfautomata,pgfheaps,pgfshade}
\usepackage{amsmath,amssymb}
\usepackage[latin1]{inputenc}
\usepackage{colortbl}
\usepackage[english]{babel}

%\usepackage{lmodern}
%\usepackage[T1]{fontenc} 

\usepackage{times}
\usepackage{multicol}
\usepackage{xspace}
\usepackage{ulem}

\usepackage{graphicx}
\graphicspath{{./figs/}}

\usepackage{listings}
\lstloadlanguages{[ISO]C++}
\lstset{language=[ISO]C++,
  texcl=true,
  columns=fullflexible,
  basicstyle={\small\sffamily}, % normal footnote small scriptsize tiny
  commentstyle=\itshape,
  showstringspaces=false,
  numberstyle=\tiny,
%  morekeywords={where, auto, concept, concept_map, axiom, late_check, final, abstract},
  morecomment=[s]{/*}{*/}
}


\setbeamercovered{dynamic}

%
% The following defintions are peculiar to this particular
% presetation. They have nothing to do with the beamer class
%

\newcommand{\Lang}[1]{\operatorname{\text{\textsc{#1}}}}

\newcommand{\Class}[1]{\operatorname{\mathchoice
  {\text{\normalfont\small #1}}
  {\text{\normalfont\small #1}}
  {\text{\normalfont#1}}
  {\text{\normalfont#1}}}}

\newcommand{\DOF}{\Class{DOF}}
\newcommand{\NOF}{\Class{NOF}}
\newcommand{\DOFpoly}{\Class{DOF}_{\operatorname{poly}}}
\newcommand{\NOFpoly}{\Class{NOF}_{\operatorname{poly}}}


\newcommand{\Nat}{\mathbb{N}}
\newcommand{\Set}[1]{\{#1\}}



\title[Milena: A Tutorial]
{Milena: A Tutorial}

\author[Milena Team]{Milena Team}

\institute[LRDE]{EPITA Research and Development Laboratory (LRDE)}

\date[EPITA-LRDE 2007]{November 2007}

\subject{}

%\pgfdeclaremask{lrde}{lrde-logo-mask}
%\pgfdeclareimage[mask=lrde,width=0.6cm]{lrde-logo}{lrde}
%\pgfdeclareimage[width=0.6cm]{lrde-logo}{lrde}

%\logo{\vbox{\hbox to 1cm{\hfil\pgfuseimage{lrde-logo}}}}
\logo{}



%###########################################################
% NEW!
%###########################################################

\newcommand{\fal}{$\bullet$\xspace}
\definecolor{darkgreen}{rgb}{0.1,0.7,0.1}

%###########################################################
% end of NEW!
%###########################################################



\begin{document}

\frame{\titlepage}

\section<presentation>*{Outline}

\begin{frame}
  \frametitle{Outline}
{\scriptsize
  \tableofcontents[part=1,pausesections]
}
\end{frame}

\AtBeginSubsection[]
{
  \begin{frame}<beamer>
    \frametitle{Outline}
{\scriptsize
    \tableofcontents[current,currentsubsection]
}
  \end{frame}
}

\part<presentation>{Main Talk}




% macros

\newcommand{\oln}{\textsc{Olena}\xspace}
\newcommand{\mln}{\textsc{Milena}\xspace}

% end of macros



%############################################################
\section{Part 1: Introduction}



%============================================================
\subsection{What is \mln?}


%........................................................................
\begin{frame}
  \frametitle{What is \oln?}

  \oln is the name for
  \begin{itemize}
  \item the project of building some modern image processing tools
  \item the platform, including
    \begin{itemize}
    \item a library
    \item command line executables
    \item some documentation
    \item etc.
    \end{itemize}
  \end{itemize}
  
  \bigskip
  
  \begin{block}{\mln}
    \mln is the C++ image processing\footnote{In the following, IP is
      ``Image Processing'' for short.} library of \oln.
  \end{block}
  
\end{frame}



%........................................................................
\begin{frame}
  \frametitle{Yet Another Image Processing Library (YAIPL) ?}

  \begin{block}{Yes!}
    \begin{itemize}
    \item Many libraries exist that can fulfill one's needs.
    \item If you're happy with your favorite tool, we cannot force you
      to change for \mln...
    \item Though, you might have a look at \mln and being seduced!
    \end{itemize}
    
  \end{block}
 
  \bigskip

  \begin{block}{No!}
    \begin{itemize}
    \item \mln is rather different that available libraries.
    \item A lot of convenient data structures that \emph{really} helps
      you in developing IP solutions.
    \end{itemize}
  \end{block}

\end{frame}


%........................................................................
\begin{frame}
  \frametitle{A Short History of the \oln Project}

  {\small
    
  \begin{itemize}
  \item 2000: Start of the project.
  \item From Nov. 2001 to April 2004: Evolution from version 0.1 to 0.10.\\
    The level of genericity we expected from the lib was partially obtained...
  \item February 2007: Update to conform modern C++ compilers = version 0.11.\\
  \end{itemize}

  \medskip

  \textit{During 3 years we developed a prototype to experiment with
    genericity and to try to meet our objectives.}

  \medskip

  \begin{itemize}
  \item From June 2007 up to new: Re-writing of the library with a
    programming paradigm that rocks.
  \end{itemize}

}%small

\end{frame}



%============================================================
\subsection{Features of the \mln Library}

%........................................................................
\begin{frame}
  \frametitle{Feature List}

  \begin{itemize}
  \item Generic...
  \item Efficient so that one can process large images.
  \item Quite as easy to use than a C or Java library.
  \item Many tools to help writing readable algorithms in a concise way.
  \end{itemize}

\end{frame}


%........................................................................
\begin{frame}
  \frametitle{Genericity}

  \begin{block}{\mln is generic}
    \begin{itemize}
    \item Put shortly it works on various types of images.
    \item Algorithms are highly reusable.
    \end{itemize}
  \end{block}

% FIXME: Say more.

\end{frame}


%........................................................................
\begin{frame}
  \frametitle{Efficiency}

  \begin{block}{\mln is efficient}
    \begin{itemize}
    \item Written in C++ without the cost of function calls.
    \item Specialized algorithms are provided.
    \end{itemize}
  \end{block}

% FIXME: Say more.

\end{frame}


%........................................................................
\begin{frame}
  \frametitle{Easy to Use}

  \begin{block}{\mln is easy to use}
    \begin{itemize}
    \item Just slightly more difficult to use than a library in C or Java.
    \item The user mainly write procedure calls.
    \end{itemize}
  \end{block}

% FIXME: Say more.

\end{frame}


%........................................................................
\begin{frame}
  \frametitle{Many Tools}

  \begin{block}{\mln provides many tools}
    \begin{itemize}
    \item A maximal amount of work is saved for the user.
    \item Claim: you do not think that IP people are ready to add tools to a lib.
    \end{itemize}
  \end{block}

% FIXME: Say more.

\end{frame}




%============================================================
\subsection{Getting Started with \mln}


%........................................................................
\begin{frame}
  \frametitle{What Is Needed}

  \begin{itemize}
  \item A C++ compiler (\texttt{g++-4} is great and fast).
  \item A browser (e.g., \texttt{Firefox})
  \item A pdf reader (e.g., \texttt{kpdf})
  \item Either \texttt{unzip} or (\texttt{gzip} and \texttt{tar})
  \item A directory to uncompress the \mln archive.
  \end{itemize}

\end{frame}


%........................................................................
\begin{frame}[fragile]
  \frametitle{Installation}

  \begin{enumerate}
    \item Get a snapshot of \mln from the web\\
      {\scriptsize \url{http://olena.lrde.epita.fr/}}
    \item Uncompress the archive.
    \item Have a look. % FIXME
  \end{enumerate}

  For instance:

{\tiny
\begin{verbatim}
tegucigalpa% cd
tegucigalpa% mkdir milena
tegucigalpa% cd milena
tegucigalpa% mv /tmp/milena-1.0-alpha.tar.gz .
tegucigalpa% tar zxvf *
tegucigalpa% ls doc
tegucigalpa% ls mln
\end{verbatim}
}

\end{frame}



%........................................................................
\begin{frame}
  \frametitle{The Main Directories}

  \begin{center}
    \begin{tabular}{ll}
      \texttt{doc}  & some documentation materials \\
      \texttt{img}  & few tiny images to play with \\
      \texttt{demo} & several examples of what can be done with \mln \\
      \texttt{mln}  & the library \\
    \end{tabular}
  \end{center}

\end{frame}



%........................................................................
\begin{frame}
  \frametitle{\mln Documentation}

FIXME

\end{frame}



%........................................................................
\begin{frame}
  \frametitle{\mln Brief Overview of the Library Contents}

  In \texttt{mln}:
\smallskip

{\scriptsize
\hspace*{-5mm}
    \begin{tabular}{|ll|ll|}
      \hline
\texttt{accu} & accumulator objects &
\texttt{arith} & arithmetical operators \\
\texttt{border} & routines about virtual border &
\texttt{canvas} & canvases \\
\texttt{convert} & conversions routines &
\texttt{core} & the library core \\
\texttt{debug} & debugging tools &
\texttt{display} & display tools \\
\texttt{draw} & drawing routines &
\texttt{estim} & estimation operators \\
\texttt{fun} & functions &
\texttt{geom} & geometrical routines \\
\texttt{histo} & histogram-related tools &
\texttt{io} & input/output routines \\
\texttt{labeling} & labeling algorithms &
\texttt{level} & point-wise operators on levels \\
\texttt{linear} & linear operators &
\texttt{literal} & definitions of literals \\
\texttt{logical} & logical operators &
\texttt{make} & routines to make objects \\
\texttt{math} & mathematical functions &
\texttt{metal} & static hard-core (metallic) tools \\
\texttt{morpho} & mathematical morphology &
\texttt{norm} & norms and related distances \\
\texttt{pw} & tools to point-wise expressions  &
\texttt{set} & mathematical set routines \\
\texttt{tag} & some tags &
\texttt{test} & testing routines \\
\texttt{trace} & tracing helpers &
\texttt{trait} & definitions of traits \\
\texttt{util} & miscellaneous utilities &
\texttt{value} & types of values \\
\texttt{win} & windows & & \\
      \hline
    \end{tabular}
} % scriptsize

\end{frame}



%........................................................................
\begin{frame}[fragile]
  \frametitle{Illustration}


\begin{minipage}{0.5\linewidth}
{\it Algorithm:}\\
\smallskip
{\tiny
$\forall p \in \mathcal{D}(f), \;\; \mathit{oper}(f(p), \, c)$
}

\bigskip\medskip

{\it Olena in 2007:}
\smallskip
\begin{lstlisting}[basicstyle={\tiny\sffamily}]
template <typename O, typename I, typename T>
void op(Image<I>& f, T c)
{
  O oper;
  oln_piter(I) p(f.domain());
  for_all(p)
    oper(f(p), c);
}

\end{lstlisting}

\end{minipage}
  %
  \hspace*{2mm}
  % 
\begin{minipage}{.45\linewidth}

{\it Olena in 2000:}
\smallskip
\begin{lstlisting}[basicstyle={\tiny\sffamily}]
template< typename O,
          template< class U > class get_A = get_value,
          typename P = Pred_true >
struct op
{
  template< typename I > static
  void on( I& f,
           const get_A< I::value_type >::output_type& c,
           P pred = P() )
    {
      O oper;
      get_A< I::value_type > access;
      I::iterator_type  iter( f );
      for ( iter.first(); ! iter.isDone(); iter.next() )
          if ( pred( access( iter() ) ) )
            oper( access( iter() ), c );
    }
};
\end{lstlisting}
  
\end{minipage}

\end{frame}



%############################################################
\end{document}

%%% Local Variables:
%%% mode: latex
%%% eval: (ispell-change-dictionary "american")
%%% TeX-master: t

% LocalWords: IP interoperability morphers Milena
