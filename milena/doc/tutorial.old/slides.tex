% $Header: /cvsroot/latex-beamer/latex-beamer/examples/beamerexample1.tex,v 1.46 2004/10/07 20:53:07 tantau Exp $

\documentclass{beamer}
%\usepackage{beamerthemesplit}

%\documentclass{article}
%\usepackage[envcountsect]{beamerarticle}

% Do NOT take this file as a template for your own talks. Use a file
% in the directory solutions instead. They are much better suited.

% Try the class options [notes], [notes=only], [trans], [handout],
% [red], [compress], [draft] and see what happens!

% Copyright 2003 by Till Tantau <tantau@users.sourceforge.net>.
%
% This program can be redistributed and/or modified under the terms
% of the LaTeX Project Public License Distributed from CTAN
% archives in directory macros/latex/base/lppl.txt.

% For a green structure color use:
%\colorlet{structure}{green!50!black}

 \mode<article> % only for the article version
 {
   \usepackage{fullpage}
   \usepackage{myhyperref}
 }


 \mode<presentation>
 {
   \setbeamertemplate{background canvas}[vertical shading][bottom=red!10,top=blue!10]
   \usetheme{Warsaw}
   \usefonttheme[onlysmall]{structurebold}


% Put the page number too.
\defbeamertemplate*{footline}{infolines theme without institution}
{
  \leavevmode%
  \hbox{%
  \begin{beamercolorbox}[wd=.333333\paperwidth,ht=2.25ex,dp=1.125ex,center]{author in head/foot}%
    \usebeamerfont{author in head/foot}\insertshortauthor
  \end{beamercolorbox}%
  \begin{beamercolorbox}[wd=.333333\paperwidth,ht=2.25ex,dp=1.125ex,center]{title in head/foot}%
    \usebeamerfont{title in head/foot}\insertshorttitle
  \end{beamercolorbox}%
  \begin{beamercolorbox}[wd=.333333\paperwidth,ht=2.25ex,dp=1.125ex,right]{date 
in head/foot}%
    \usebeamerfont{date in head/foot}\insertshortdate{}\hspace*{2em}
    \insertframenumber{} / \inserttotalframenumber\hspace*{2ex}
  \end{beamercolorbox}}%
  \vskip0pt%
}


}
% end of mode presentation



%\setbeamercolor{math text}{fg=green!50!black}
%\setbeamercolor{normal text in math text}{parent=math text}

\usepackage{pgf,pgfarrows,pgfnodes,pgfautomata,pgfheaps,pgfshade}
\usepackage{amsmath,amssymb}
\usepackage[latin1]{inputenc}
\usepackage{colortbl}
\usepackage[english]{babel}

%\usepackage{lmodern}
%\usepackage[T1]{fontenc} 

\usepackage{times}
\usepackage{multicol}
\usepackage{xspace}
\usepackage{ulem}

\usepackage{graphicx}
\graphicspath{{./figs/}}

\usepackage{listings}
\lstloadlanguages{[ISO]C++}
\lstset{language=[ISO]C++,
  texcl=true,
  columns=fullflexible,
  basicstyle={\scriptsize\sffamily}, % normal footnote small scriptsize tiny
  commentstyle=\itshape,
  showstringspaces=false,
  numberstyle=\tiny,
%  morekeywords={where, auto, concept, concept_map, axiom, late_check, final, abstract},
  morecomment=[s]{/*}{*/}
}


\setbeamercovered{dynamic}

%
% The following defintions are peculiar to this particular
% presetation. They have nothing to do with the beamer class
%

\newcommand{\Lang}[1]{\operatorname{\text{\textsc{#1}}}}

\newcommand{\Class}[1]{\operatorname{\mathchoice
  {\text{\normalfont\small #1}}
  {\text{\normalfont\small #1}}
  {\text{\normalfont#1}}
  {\text{\normalfont#1}}}}

\newcommand{\DOF}{\Class{DOF}}
\newcommand{\NOF}{\Class{NOF}}
\newcommand{\DOFpoly}{\Class{DOF}_{\operatorname{poly}}}
\newcommand{\NOFpoly}{\Class{NOF}_{\operatorname{poly}}}


\newcommand{\Nat}{\mathbb{N}}
\newcommand{\Set}[1]{\{#1\}}



\title[Milena: A Tutorial---Part 1]
{Milena: A Tutorial---Part 1}

\author[Milena Team]{Milena Team}

\institute[LRDE]{EPITA Research and Development Laboratory (LRDE)}

\date[EPITA-LRDE 2007]{November 2007}

\subject{}

%\pgfdeclaremask{lrde}{lrde-logo-mask}
%\pgfdeclareimage[mask=lrde,width=0.6cm]{lrde-logo}{lrde}
%\pgfdeclareimage[width=0.6cm]{lrde-logo}{lrde}

%\logo{\vbox{\hbox to 1cm{\hfil\pgfuseimage{lrde-logo}}}}
\logo{}



%###########################################################
% NEW!
%###########################################################

\newcommand{\fal}{$\bullet$\xspace}
\definecolor{darkgreen}{rgb}{0.1,0.7,0.1}


\newcommand{\cpp}{{C++}\xspace}

\newcommand{\kw}[1]{{\color{blue}{\texttt{#1}}}\xspace}
\newcommand{\code}[1]{{\scriptsize{\texttt{#1}}}\xspace}
\newcommand{\var}[1]{\texttt{#1}\xspace}

\newcommand{\struct}{\kw{struct}}

\newcommand{\pointIId}{\code{point2d}}


\newcommand{\oln}{\textsc{Olena}\xspace}
\newcommand{\mln}{\textsc{Milena}\xspace}


%###########################################################
% end of NEW!
%###########################################################



\begin{document}

\frame{\titlepage}

\section<presentation>*{Outline}

\begin{frame}
  \frametitle{Outline}
{\scriptsize
  \tableofcontents[part=1,pausesections]
}
\end{frame}

\AtBeginSubsection[]
{
  \begin{frame}<beamer>
    \frametitle{Outline}
{\scriptsize
    \tableofcontents[current,currentsubsection]
}
  \end{frame}
}

\part<presentation>{Main Talk}





%############################################################
\section{About \oln and \mln}



%============================================================
\subsection{What is \mln?}


%........................................................................
\begin{frame}
  \frametitle{What is \oln?}

  \oln is the name for
  \begin{itemize}
  \item the project of building some modern image processing tools
  \item the platform, including
    \begin{itemize}
    \item a library
    \item command line executables
    \item some documentation
    \item etc.
    \end{itemize}
  \end{itemize}
  
  \bigskip
  
  \begin{block}{\mln}
    \mln is the C++ image processing\footnote{In the following, IP is
      ``Image Processing'' for short.} library of \oln.
  \end{block}
  
\end{frame}



%........................................................................
\begin{frame}
  \frametitle{Yet Another Image Processing Library (YAIPL) ?}

  \begin{block}{Yes!}
    \begin{itemize}
    \item Many libraries exist that can fulfill one's needs.
    \item If you're happy with your favorite tool, we cannot force you
      to change for \mln...
    \item Though, you might have a look at \mln and be seduced!
    \end{itemize}
    
  \end{block}
 
  \bigskip

  \begin{block}{No!}
    \begin{itemize}
    \item \mln is rather different than available libraries.
    \item A lot of convenient data structures that \emph{really} help
      you in developing IP solutions.
    \end{itemize}
  \end{block}

\end{frame}


%........................................................................
\begin{frame}
  \frametitle{A Short History of the \oln Project}

  {\small
    
  \begin{itemize}
  \item 2000: Start of the project.
  \item From Nov. 2001 to April 2004: Evolution from version 0.1 to 0.10.\\
    The level of genericity we expected from the lib was partially obtained...
  \item February 2007: Update to conform modern C++ compilers = version 0.11.\\
  \end{itemize}

  \medskip

  \textit{During those 3 years we developed a prototype to experiment with
    genericity and to try to meet our objectives.}

  \medskip

  \begin{itemize}
  \item From June 2007 up to now: Re-writing of the library with a
    programming paradigm that rocks.
  \end{itemize}

}%small

\end{frame}



%........................................................................
\begin{frame}[fragile]
  \frametitle{Illustration of the Evolution}


\begin{minipage}{0.5\linewidth}
{\it Algorithm:}\\
\smallskip
{\tiny
$\forall p \in \mathcal{D}(f), \;\; f(p) = \mathit{h}(f(p))$
}

\bigskip\medskip

{\it In 2007:}
\smallskip
\begin{lstlisting}[basicstyle={\tiny\sffamily}]
template <typename I, typename H>
void transform_inplace(Image<I>& f_,
                       const Function_v2v<H>& h_)
{
  I& f = exact(f);
  const H& h = exact(h_);
  mln_piter(I) p(f.domain());
  for_all(p)
    f(p) = h(f(p));
}

\end{lstlisting}

\end{minipage}
  %
  \hspace*{2mm}
  % 
\begin{minipage}{.45\linewidth}

{\it The same code in 2000:}
\smallskip
\begin{lstlisting}[basicstyle={\tiny\sffamily}]
template< typename H,
          template< class U > class get_A = get_value,
          typename P = Pred_true >
struct transform_inplace
{
  template< typename I > static
  void on( I& f,
           P pred = P() )
    {
      H h;
      get_A< I::value_type > access;
      I::iterator_type  iter( f );
      for ( iter.first(); ! iter.isDone(); iter.next() )
          if ( pred( access( iter() ) ) )
            access( iter() ) = h( access( iter() ) );
    }
};
\end{lstlisting}
  
\end{minipage}

\end{frame}



%============================================================
\subsection{Features of the \mln Library}


%........................................................................
\begin{frame}%[<+->]
  \frametitle{What's In a Library}

  \begin{itemize}
  \item algorithms:\\
    procedures dedicated to image processing and pattern recognition
    \smallskip
    % 
  \item data types for pixel values:\\
    e.g., gray level types, color types
    \smallskip
    % 
  \item data structures:\\
    image types or point set types for instance
    \smallskip
    % 
  \item auxiliary tools...
  \end{itemize}

\end{frame}


%........................................................................
\begin{frame}
  \frametitle{\mln as a Feature List}

  \begin{itemize}
  \item Generic...
  \item Efficient so that one can process large images.
  \item Almost as easy to use as a C or Java library.
  \item Many tools to help writing readable algorithms in a concise way.
  \end{itemize}

\end{frame}


%........................................................................
\begin{frame}
  \frametitle{Genericity}

  \begin{block}{\mln is generic}
    \begin{itemize}
    \item Put shortly it works on various types of images.
    \item Algorithms are highly reusable.
    \end{itemize}
  \end{block}

% FIXME: Say more.

\end{frame}


%........................................................................
\begin{frame}
  \frametitle{Efficiency}

  \begin{block}{\mln is efficient}
    \begin{itemize}
    \item Written in C++ without the cost of function calls.
    \item Specialized algorithms are provided.
    \end{itemize}
  \end{block}

% FIXME: Say more.

\end{frame}


%........................................................................
\begin{frame}
  \frametitle{Easy to Use}

  \begin{block}{\mln is easy to use}
    \begin{itemize}
    \item Just slightly more difficult to use than a library in C or Java.
    \item The user mainly write routine calls.
    \end{itemize}
  \end{block}

% FIXME: Say more.

\end{frame}


%........................................................................
\begin{frame}
  \frametitle{Many Tools}

  \begin{block}{\mln provides many tools}
    \begin{itemize}
    \item A maximal amount of work is saved for the user.
    \item Claim: you do not think that IP people are ready to add tools to a lib.
    \end{itemize}
  \end{block}

% FIXME: Say more.

\end{frame}




%============================================================
\subsection{Getting Started with \mln}


%........................................................................
\begin{frame}
  \frametitle{What Is Needed}

  \begin{itemize}
  \item A C++ compiler (\texttt{g++-4} is great and fast).
  \item A browser (e.g., \texttt{Firefox})
  \item A pdf reader (e.g., \texttt{kpdf})
  \item Either \texttt{unzip} or (\texttt{gzip} and \texttt{tar})
  \item A directory to uncompress the \mln archive.
  \end{itemize}

\end{frame}


%........................................................................
\begin{frame}[fragile]
  \frametitle{Installation}

  \begin{enumerate}
    \item Get a snapshot of \mln from the web\\
      {\scriptsize \url{http://olena.lrde.epita.fr/}}
    \item Uncompress the archive.
    \item Have a look. % FIXME
  \end{enumerate}

  For instance:

{\tiny
\begin{verbatim}
tegucigalpa% cd
tegucigalpa% mkdir milena
tegucigalpa% cd milena
tegucigalpa% mv /tmp/milena-1.0-alpha.tar.gz .
tegucigalpa% tar zxvf *
tegucigalpa% ls doc
tegucigalpa% ls mln
\end{verbatim}
}

\end{frame}



%........................................................................
\begin{frame}
  \frametitle{The Main Directories}

  \begin{center}
    \begin{tabular}{ll}
      \texttt{doc}  & some documentation materials \\
      \texttt{img}  & few tiny images to play with \\
      \texttt{demo} & several examples of what can be done with \mln \\
      \texttt{mln}  & the library \\
    \end{tabular}
  \end{center}

\end{frame}



% %........................................................................
% \begin{frame}
%   \frametitle{\mln Documentation}

% FIXME

% \end{frame}



%........................................................................
\begin{frame}
  \frametitle{\mln Brief Overview of the Library Contents}

  In \texttt{mln}:
\smallskip

{\scriptsize
\hspace*{-5mm}
    \begin{tabular}{|ll|ll|}
      \hline
\texttt{accu} & accumulator objects &
\texttt{arith} & arithmetical operators \\
\texttt{border} & routines about virtual border &
\texttt{canvas} & canvases \\
\texttt{convert} & conversion routines &
\texttt{core} & the library core \\
\texttt{debug} & debugging tools &
\texttt{display} & display tools \\
\texttt{draw} & drawing routines &
\texttt{estim} & estimation operators \\
\texttt{fun} & functions &
\texttt{geom} & geometrical routines \\
\texttt{histo} & histogram-related tools &
\texttt{io} & input/output routines \\
\texttt{labeling} & labeling algorithms &
\texttt{level} & point-wise operators on levels \\
\texttt{linear} & linear operators &
\texttt{literal} & definitions of literals \\
\texttt{logical} & logical operators &
\texttt{make} & routines to make objects \\
\texttt{math} & mathematical functions &
\texttt{metal} & static hard-core (metallic) tools \\
\texttt{morpho} & mathematical morphology &
\texttt{norm} & norms and related distances \\
\texttt{pw} & tools to point-wise expressions  &
\texttt{set} & mathematical set routines \\
\texttt{tag} & some tags &
\texttt{test} & testing routines \\
\texttt{trace} & tracing helpers &
\texttt{trait} & definitions of traits \\
\texttt{util} & miscellaneous utilities &
\texttt{value} & types of values \\
\texttt{win} & windows & & \\
      \hline
    \end{tabular}
} % scriptsize

\end{frame}




%........................................................................
\begin{frame}%[<+->]
  \frametitle{Why Choosing \mln?}

  \begin{itemize}
  \item rather different...
    \smallskip
    % 
  \item a strong potential
    \smallskip
    % 
  \item you want to focus on what you do,\\
    not on implementation details about how to do it
    \smallskip
    % 
  \item you have not yet found a library to easily process your
    particular types of data
  \end{itemize}

\end{frame}



%........................................................................
\begin{frame}%[<+->]
  \frametitle{Dev status}

  \begin{itemize}
  \item alpha
  \item some cleanings remain to be done
  \item an intensive test phase is upcoming...
  \item rough documentation (yet in progress)
  \end{itemize}

\end{frame}



%############################################################"
\section{A Short Tour of \cpp}

%........................................................................
\begin{frame}
  \frametitle{The Running Example}

  A class that represents:
  \begin{itemize}
  \item a discrete point of the 2D plane
  \item a node of a square grid
  \item a point of a ``classical'' 2D image 
  \item basically a couple of integer coordinates
  \item namely \pointIId
  \end{itemize}

\end{frame}


%============================================================
\subsection{Very Basic Notions}

%........................................................................
\begin{frame}
  \frametitle{Class v. Object}

  \begin{block}{Class}
    A class is a type that describes at the same time both data and behavior.
    \smallskip
    % 

    \begin{itemize}
    \item the data are described by attributes\\
      (equiv.) structure fields of a \struct
      \smallskip
      % 
    \item the behavior is described by methods\\
      (equiv.) procedures/functions attached to the class
    \end{itemize}

  \end{block}

\end{frame}


%........................................................................
\begin{frame}
  \frametitle{Object v. Class}

  \begin{block}{Object}
    An object is an instance of a class.
    \smallskip
    % 

    \begin{itemize}
    \item the object data are a set of values: the state of the object
      \smallskip
      % 
    \item the object behavior is what happens at run-time when a
      method is called on that particular object
    \end{itemize}
  \end{block}

\end{frame}


%........................................................................
\begin{frame}
  \frametitle{Getting an Object}

  \begin{block}{Constructor}
    A constructor is a special method to instantiate a class / to get
    an object.

    \begin{itemize}
    \item it allows initializing the state of this object
    \end{itemize}
  \end{block}

\end{frame}


%........................................................................
\begin{frame}[fragile]
  \frametitle{Example}

  \begin{itemize}
  \item \pointIId is a class
  \item \var{p} is a variable that represents a point object
  \item this variable designates one particular 2D point\\
    at a given couple of coordinates: row and column.
  \end{itemize}

\smallskip

\begin{lstlisting}
point2d p(5,1); // construction of an object
std::cout << p << std::endl; // print this object on the std output
\end{lstlisting} % >>

gives: \code{(5,1)}

meaning that \var{p} represents the point of the 2D grid located at
row 5 and column 1.

\end{frame}



%........................................................................
\begin{frame}[fragile]
  \frametitle{Accessing the State of an Object}

  \begin{itemize}
  \item data are usually protected / hidden from the user
  \item reading and modifying them is performed through method calls
  \end{itemize}

\smallskip

Here is a call to the method  \code{row()} defined in the \pointIId class:

\begin{lstlisting}
std::cout << p.row() << std::endl; // gives: 5
\end{lstlisting} % >>

\var{p} is the object targeted by the method call: we want to print
the row of this particular point.

\end{frame}


%........................................................................
\begin{frame}[fragile]
  \frametitle{Methods are just like C functions!}

The previous example is just the \cpp equivalent of this C code:

\begin{lstlisting}
#include <stdio.h>

struct point2d {
  int row, col;
};

int get_row(struct point2d p) {
  return p.row;
}

int main() {
  struct point2d p;
  printf("%d\n", get_row(p));
  return 0;
}
\end{lstlisting}

\end{frame}



%........................................................................
\begin{frame}[fragile]
  \frametitle{Modifying the State of an Object (1/2)}

modifying an object is performed through method calls:

\begin{lstlisting}
p.row() = 7; // method call
std::cout << p.row() << std::endl; // now gives 7
\end{lstlisting} % >>

\smallskip

the C equivalent of this method would be:
\begin{lstlisting}
void set_row(struct point2d* p, int r) {
  assert(p != 0);
  p->row = r;
}

int main() {
  struct point2d p;
  set_row(&p, 7);
  return 0;
}
\end{lstlisting}

note that \code{p.row() = 7} looks more natural than \code{set\_row(\&p, 7)}.

\end{frame}


%........................................................................
\begin{frame}[fragile]
  \frametitle{Modifying the State of an Object (2/2)}

  accessing and modifying through method calls allow for some control:
  \begin{itemize}
  \item one cannot do anything with an object
  \item especially putting it in an invalid state
  \end{itemize}

  \smallskip

imagine that \var{ima} is a 3$\times$3 image (starting from (0,0))

trying to access the image value at point \var{p}, like with:
\begin{lstlisting}
std::cout << ima(p) << std::endl; // remind that p is at a row 7...
\end{lstlisting} % >>
will hopefully produce an error at run-time!

\end{frame}



%........................................................................
\begin{frame}
  \frametitle{Operators}

  In \cpp some operators can be re-defined to get a high
  expressiveness in client code.

  \begin{itemize}
  \item ``\code{ima(p)}'' is a call to the parenthesis operator
    defined as a method in every image class
    \begin{itemize}
    \item that's great, an image looks like a function from points to
      values
    \end{itemize}
    \smallskip
    % 
  \item \code{2 * p} calls a multiplication operator defined as a
    procedure (function):
    \begin{itemize}
    \item this way one can easily use arithmetics over points
    \item the result is a 2D point which coordinates are twice those of \var{p}
    \end{itemize}
  \end{itemize}

  Operators are very convenient!

\end{frame}



%============================================================
\subsection{References}


%........................................................................
\begin{frame}
  \frametitle{Reference v. Pointer}

  A reference looks like a pointer, yet:
  \begin{itemize}
  \item without the pointer notations
    \begin{itemize}
    \item no need to take the address (with \&) of an object
    \item no pointer arithmetics
    \item no $->$ to access members
    \end{itemize}
  \item it \emph{always} designates the same object
    \begin{itemize}
    \item one can reuse a pointer and make it points elsewhere, that's
      not the case for a reference
    \item it is like a ``constant pointer''\\
      ``\code{int*const}'', not ``\code{int*}''
    \item it has to be initialized
    \end{itemize}
  \end{itemize}

\end{frame}


%........................................................................
\begin{frame}[fragile]
  \frametitle{Example (1/2)}

This C program:

\begin{lstlisting}
void set_row(point2d* p, int r) {
  p->row = r;
}
// used with:
point2d p;
set_row(&p, 7);
\end{lstlisting}

\smallskip

can be re-written as:
\begin{lstlisting}
void set_row(point2d& p, int r) {
  p.row = r; // no ``$->$'' here
}
// used with:
point2d p;
set_row(p, 7);
\end{lstlisting}

\end{frame}


%........................................................................
\begin{frame}[fragile]
  \frametitle{Example (1/2)}

and this one:
\begin{lstlisting}
int get_row(point2d p) { // copy of a point at procedure call
  return p.row;
}
\end{lstlisting}

\smallskip

is better written as:
\begin{lstlisting}
int get_row(const point2d& p) {
  return p.row;
}
// used with:
point2d p;
int i = get_row(p);
\end{lstlisting}
which avoids the copy of a point at function call.

\end{frame}


%........................................................................
\begin{frame}[fragile]
  \frametitle{References rock!}

Realize that with:
\begin{lstlisting}
class point2d {
public:
  int& row() { return row_; }
  // ...
private:
  int row_, col_;
};
\end{lstlisting}

one can write:
\begin{lstlisting}
point2d p;
p.row() = 5;
\end{lstlisting}

so it really performs \code{p.row\_ = 5}

\end{frame}


%........................................................................
\begin{frame}[fragile]
  \frametitle{A few Remarks}

  \begin{itemize}
  \item the attribute \code{row\_} of the class \code{point2d} is not
    accessible from the user
    \begin{itemize}
    \item thanks to the keyword \kw{private}
    \item writing \code{p.row\_} outside this class is not allowed (does
      not compile)
    \end{itemize}
    \smallskip
    % 
  \item the method \code{row()} is accessible (keyword \kw{public})
    \begin{itemize}
    \item in the method body we have some room to add code
    \item a simple access to data can perform some clever stuff that
      you do not really have to know (neither want to)!
    \end{itemize}
  \end{itemize}

\end{frame}




%============================================================
\subsection{Inheritance}


%........................................................................
\begin{frame}[fragile]
  \frametitle{The ``Is-A'' Relationship}

  \begin{block}{Inheritance}
    The inheritance between classes maps the ``is-a'' relationship.
  \end{block}

For instance, since we can say that a rabbit is an animal:
\begin{itemize}
\item it is safe to make the \code{rabbit} class inherits from the
  \code{animal} one
\item we also say that:
  \begin{itemize}
  \item \code{rabbit} derives from \code{animal}
  \item \code{animal} is a \emph{base} class for \code{rabbit}
  \end{itemize}
\end{itemize}

In \cpp we write:
\begin{lstlisting}
class animal { ... };
class rabbit : public animal { ... };
\end{lstlisting}

\end{frame}



%........................................................................
\begin{frame}[fragile]
  \frametitle{Using the ``Is-A'' Relationship}

  When one wants to have a procedure to feed an animal, one can write:

\begin{lstlisting}
void feed(animal& a) {
  ...
}
\end{lstlisting}

then the following use is valid 
\begin{lstlisting}
int main() {
  rabbit r;
  feed(r); // works fine since a rabbit ``is-an'' animal
  ...
}
\end{lstlisting}

\end{frame}



%........................................................................
\begin{frame}[fragile]
  \frametitle{One Object but Several Variables and Types (1/2)}

Considering \emph{only} the \code{feed} routine:
\begin{lstlisting}
void feed(animal& a) {
  ...
}
\end{lstlisting}

\smallskip

we can say that:
\begin{itemize}
\item the variable \code{a} can represent an object being of any type
  deriving from \code{animal}
\item it may be a \code{rabbit}
  \begin{itemize}
  \item yet we do not really know!
  \item it might be a \code{sheep} instead...
  \end{itemize}
\end{itemize}

\smallskip

\begin{center}
  This routine is \emph{general} since it can work on objects of
  different types.
\end{center}

\end{frame}



%........................................................................
\begin{frame}[fragile]
  \frametitle{One Object but Several Variables and Types (2/2)}

Now considering the entire program, with:
\begin{lstlisting}
int main() {
  rabbit r;
  feed(r); // first call
  sheep s;
  feed(s); // another call
}
\end{lstlisting}

\begin{itemize}
\item the variable \code{r} represents an object whose type
  precisely \code{rabbit}
  \begin{center}
    we say that it is the \emph{exact} type behind this variable
  \end{center}
  %
\item for the first call to \code{feed}, we know that \code{a} represents a rabbit
  \begin{center}
    during this execution, the \emph{exact} type of \code{a} is \code{rabbit}
  \end{center}
\end{itemize}

\end{frame}


%........................................................................
\begin{frame}
  \frametitle{Static type}

  \begin{block}{Static type}
    \begin{itemize}
      \item A variable is declared with one type.
      \item This type can be read in the code; it is known at compile-time.
      \item For instance, in ``\lstinline@animal& a@'': \code{a} is an animal.
    \end{itemize}
    
    The variable type is said to be the \emph{static} type.
  \end{block}
  
\end{frame}



%........................................................................
\begin{frame}
  \frametitle{Exact type}
  
  \begin{block}{Exact type}
    \begin{itemize}
    \item A variable represents an object.
    \item Its static type can be a \textit{base} class (like in ``\lstinline@animal& a@'')
    \item In that case
      \begin{itemize}
      \item at compile-time: there are many possible types of objects represented
      \item at run-time: there is one object represented so just one type.
      \end{itemize}
    \end{itemize}

    At run-time, the type of the object behind a variable is said to
    be the \emph{exact} type.
  \end{block}
  
\end{frame}



%........................................................................
\begin{frame}
  \frametitle{A Clue to Understand \mln}

  About ``classical'' object-orientation:

  \begin{itemize}
  \item abstractions (like \code{animal}) lead to poor
    performance at run-time when involved in intensive scientific code.
    %
  \item it is due to the fact that the exact type is lost\\
    (the \kw{virtual} keyword has an effective cost)
  \end{itemize}
  
  \smallskip

  The clue:

  \begin{itemize}
  \item genericity leads to dedicated code, thus it is efficient at run-time
    %
  \item though we really want abstractions!
  \end{itemize}
  

\end{frame}



%============================================================
\subsection{What a Class Can Contain}



%........................................................................
\begin{frame}[fragile]
  \frametitle{Methods and Attributes}

\begin{lstlisting}
class point2d {
public:
  point2d(int r, int c) { // constructor
    row_ = r; col_ = col;
  }
  int& operator[](unsigned i) {
    assert(i < 2);
    return i == 0 ? row_ : col_;
  }
  int  row() const  { return row_; }
  int& row()  { return row_; }
  // ...
private:
  int row_, col_;
}
\end{lstlisting}

Sample use:
\begin{lstlisting}
point2d p(5, 1);
assert(p[0] == p.row());
\end{lstlisting}

\end{frame}



%........................................................................
\begin{frame}[fragile]
  \frametitle{Values and Typedefs}


\begin{lstlisting}
class point2d
{
public:
  enum { dim = 2 }
  typedef int coord;
  // ...
}
\end{lstlisting}

Sample use:
\begin{lstlisting}
std::cout << point2d::dim << std::endl; // gives 2
point2d::coord c;  // \code{c} is an \code{int}
\end{lstlisting}

{\scriptsize
At first glance, that seems weird to equip this class with \code{dim}
and \code{coord} (but is is \emph{not}!)
}

\end{frame}



%........................................................................
\begin{frame}
  \frametitle{Associated Types}

\begin{block}{Associated Type}
  A typedef (type alias) defined in a class (e.g., \code{coord} in
  \code{point2d}) is called an ``\emph{associated} type.''
\end{block}

\smallskip
We have defined macros to access those types:
\begin{itemize}
\item given a type T, \code{mln\_something(T)} gives the associated
  type \code{something} defined in \code{T}
\item example of use: \code{mln\_coord(point2d)}
\item we can see \code{mln\_coord} like a function that takes a type and returns a type
\end{itemize}

\smallskip

\tiny{\code{mln\_something(T)} is for a \kw{template}d piece of
  code,\\ whereas \code{mln\_something\_(T)} is for a
  \emph{non}-\kw{template}d code}

\end{frame}



%############################################################"
\section{Genericity in \cpp}


%============================================================
\subsection{Genericity for Routines}


%........................................................................
\begin{frame}
  \frametitle{About Naming}

  \begin{center}
    \begin{tabular}{|c|c|}
      \hline
      we say & a C or \cpp user says \\
      \hline \hline 
      attribute & a field (C) or member (\cpp) \\
      procedure & function \\
      method & a member function \\
      \hline 
    \end{tabular}
  \end{center}

\smallskip

In the following:
\begin{block}{routine}
  A \emph{routine} designates either a procedure (function) or a
  method (a member function).
\end{block}

\end{frame}


%........................................................................
\begin{frame}[fragile]
  \frametitle{A rationale for Genericity}

Suppose that you want a routine that computes twice its input:
\begin{lstlisting}
int twice(int i) { return 2 * i; }
\end{lstlisting}

Suppose now that you want the ``\code{twice}'' operation to work also
with values of type \kw{float}.

\begin{itemize}
\item you can rely on \emph{overloading}
\item that is the ability of defining several versions of a function
  \begin{itemize}
  \item having the \emph{same} name
  \item but different signatures
  \end{itemize}
\end{itemize}

Precisely, you write:
\begin{lstlisting}
int   twice(int   i) { return 2 * i; }
float twice(float f) { return 2 * f; }
\end{lstlisting}

\end{frame}


%........................................................................
\begin{frame}
  \frametitle{Overloading is Limited}

This code is quite poor:

\begin{itemize}
\item it is redundant
  \begin{itemize}
  \item tedious to write (copy-paste, many lines at the end)
  \item thus error-prone
  \end{itemize}
  \smallskip
  % 
\item it is still limited to \code{int} and \code{float}
  \smallskip
  % 
\item so it is not re-usable!
  \begin{itemize}
  \item understand that ``\code{twice}'' should be able to work with
    \pointIId too
  \end{itemize}
\end{itemize}

Nevertheless overloading is great; think of \code{operator*}...

\end{frame}


%........................................................................
\begin{frame}[fragile]
  \frametitle{We Want Genericity}

Actually
\begin{center}
  for every type \code{T}, \code{twice} a value \code{t} of type
  \code{T} returns \code{2 * t} which is of type \code{T}
\end{center}

So this procedure definition looks like
\begin{lstlisting}
// ...
T twice(T t) {
  return 2 * t;
}
\end{lstlisting}
except that we have to say first what \code{T} is:
\begin{lstlisting}
template <typename T>
T twice(T t) {
  return 2 * t;
}
\end{lstlisting}

\end{frame}


%........................................................................
\begin{frame}[fragile]
  \frametitle{Syntax of Genericity}

In \lstinline@template <typename T> T twice(T t)@
%
\begin{itemize}
\item the declaration ``\code{<typename T>}'' is very similar to the
  one of the procedure argument ``\code{(T t)}''
%
\item the nature of \code{t} is \code{T}, the nature of \code{T} is
  \code{typename} (so it designates a type)
%
\item the \cpp keyword introducing a generic piece of code is
  \kw{template}
  \begin{itemize}
  \item it can be read as ``for all'' (the universal quantifier $\forall$)\\
    so you read here: ``for all $<$type T$>$, we have...''
  \item the definition (symbolized by ``...'') follows a classical
    \cpp syntax
  \end{itemize}
%
\item yet the major difference is that:
  \begin{center}
    \code{t} is valued at \emph{run-time}, whereas \code{T} is valued at compile-time
  \end{center}
\end{itemize}

\end{frame}


%........................................................................
\begin{frame}[fragile]
  \frametitle{Use of a Generic Procedure}

With
\begin{lstlisting}
int main() {
  int i = 1;
  int j = twice(i); // Calls twice with \code{T} being \kw{int}
  float pi = 3.14;
  float two_pis = twice(pi); // Calls another ``version'' of twice with \code{T} being \kw{float}
}
\end{lstlisting}


Once this program is compiled
\begin{itemize}
\item two different versions of \code{twice} cohabits:
  \begin{itemize}
  \item \code{int   twice(int   t) { return 2 * t; }} and
  \item \code{float twice(float t) { return 2 * t; }}
  \end{itemize}
\item so it is not so different than overloading
\end{itemize}

except that:
\begin{center}
  this generic definition of \code{twice} is \emph{reusable}
\end{center}

\end{frame}


%........................................................................
\begin{frame}[fragile]
  \frametitle{Generic Procedures and Reusability}

Precisely
\begin{itemize}
\item the generic definition of a procedure is written once
\item and is possibly usable in a large number of different versions
\end{itemize}

If the client wants to write this kind of use:
\begin{lstlisting}
point2d p(2,3), pp;
pp = twice(p);
std::cout << pp << std::endl; // writes (4,6)
\end{lstlisting} % >>
it also works!

\end{frame}


% HERE


%........................................................................
\begin{frame}
  \frametitle{Exercise}

Consider this mathematical parameterized function:
$$
\forall a \in \mathbb{Z}, \;
f_a: \left\{
  \begin{array}{lll}
    \mathbb{R} &\rightarrow & \mathbb{R} \\
    x & \mapsto & \cos(a x)
  \end{array}
\right.
$$

\begin{itemize}
\item how can it be translated in \cpp?
\item how can a call to such a function work?
\end{itemize}

\end{frame}



%============================================================
\subsection{Genericity for Classes}


%........................................................................
\begin{frame}[fragile]
  \frametitle{A First Generic Class}

  We want to define a class to represent couples of values, whatever
  their respective type is; it gives:

\begin{lstlisting}
template <typename T1, typename T2>
struct pair {
  T1 first;
  T2 second;
};
\end{lstlisting}

\smallskip

A sample use is:
\begin{lstlisting}
pair<float, int> c;
c.first = 3.14, c.second = 3; 
\end{lstlisting}

Another possible use is:
\begin{lstlisting}
pair<bool, point2d> c;
c.first = true, c.second = p; 
\end{lstlisting}

\end{frame}


%........................................................................
\begin{frame}[fragile]
  \frametitle{Exercise}

Consider this:
\begin{lstlisting}
template <typename T1, typename T2>
struct pair
{
  template <typename S>
  void operator*=(S scalar)
  {
    first *= scalar;
    second *= scalar;
  }
  T1 first;
  T2 second;
};
\end{lstlisting}

\smallskip

\begin{itemize}
\item explain what you see
\item then write a program to use this class
\end{itemize}

\end{frame}




%############################################################"
\section{Understanding \mln}


%........................................................................
\begin{frame}[fragile]
  \frametitle{Forewords (1/2)}

We want some arithmetics over points:
\begin{itemize}
\item a ``delta-point'' is a difference between two points
\item a point + (plus) a delta-point gives a point
\item the addition (resp. subtraction) of a couple of delta-points gives a delta-point. 
\end{itemize}

\smallskip

For instance
\begin{lstlisting}
point2d p(4, -1);
dpoint2d dp(1, 2); // \code{dpoint} is ``delta-point'' for short
std::cout << (p + dp) << std::endl; // gives (5, 1)
\end{lstlisting}% Please Emacs.  >>

\end{frame}


%........................................................................
\begin{frame}
  \frametitle{Forewords (2/2)}

Our objectives:

\begin{itemize}
\item write the \code{operator+} (resp. '\code{-}') routine corresponding to\\
  \begin{center}
    ``a \code{point2d} ~+~ a \code{dpoint2d} $~\mapsto~$ a
    \code{point2d}''
  \end{center}
  % 
\item understand that we actually want:\\
  \begin{center}
    ``any \code{point P} ~+~ a compatible \code{dpoint D} $~\mapsto~$
    a \code{point P}''
  \end{center}
  \scriptsize{for instance with \code{P} and \code{D} being
    respectively \code{point3d} and \code{dpoint3d}}
  %
\item make different versions of operators cohabit...
\end{itemize}

\end{frame}



%============================================================
\subsection{First Attempts}


%........................................................................
\begin{frame}[fragile]
  \frametitle{Overloading}

\begin{lstlisting}
point2d operator+(const point2d& p, const dpoint2d& dp)
{
  point2d q(p.row() + dp.row(), p.col() + dp.col());
  return q;
}
point3d operator+(const point3d& p, const dpoint3d& dp)
{
  point3d q;
  for (unsigned i = 0; i < 3; ++i) q[i] = p[i] + dp[i];
  return q;
}
\end{lstlisting}

What do you think of that?

\end{frame}



%........................................................................
\begin{frame}[fragile]
  \frametitle{A Solution with Genericity}

\begin{lstlisting}
template <typename P, typename D>
P operator+(const P& p, const D& dp)
{
  P q;
  for (unsigned i = 0; i < P::dim; ++i)
    q[i] = p[i] + dp[i];
  return q;
}
\end{lstlisting}

\end{frame}


%........................................................................
\begin{frame}[fragile]
  \frametitle{Continuing with Genericity}

Now the subtraction:

\begin{lstlisting}
template <typename P>
? operator-(const P& p1, const P& p2)
{
  ? dp;
  for (unsigned i = 0; i < P::dim; ++i)
    dp[i] = p1[i] - p2[i];
  return dp;
}
\end{lstlisting}

What shall we write instead of the question mark?

\end{frame}


%........................................................................
\begin{frame}[fragile]
  \frametitle{Solution}

\begin{lstlisting}
template <typename P>
mln_dpoint(P) operator-(const P& p1, const P& p2)
{
  mln_dpoint(P) dp;
  for (unsigned i = 0; i < P::dim; ++i)
    dp[i] = p1[i] - p2[i];
  return dp;
}
\end{lstlisting}

\end{frame}


%........................................................................
\begin{frame}[fragile]
  \frametitle{Make Things Better}

In the addition:

\begin{lstlisting}
template <typename P, typename D>
P operator+(const P& p, const D& dp)
{
  P q;
  // accessing the dimension is really useful:
  for (unsigned i = 0; i < P::dim; ++i)
    q[i] = p[i] + dp[i];
  return q;
}
\end{lstlisting}

How can we ensure that the delta-point type \code{D} really corresponds
to \code{P}?  {\scriptsize (we really do not want \code{P} and
  \code{D} resp. being \code{point3d} and \code{dpoint2d}!)}

\end{frame}


%........................................................................
\begin{frame}[fragile]
  \frametitle{Solution}

\begin{lstlisting}
template <typename P>
P operator+(const P& p, const mln_dpoint(P)& dp)
{
  ...
}
\end{lstlisting}

\end{frame}



%........................................................................
\begin{frame}[fragile]
  \frametitle{Cohabitation is Hard}

Consider:

\begin{lstlisting}
template <typename D>
D operator+(const D& dp1, const D& dp2) {
  ... // addition of a couple of delta-points
}

template <typename I>
I operator+(const I& ima1, const I& ima2) {
  ... // addition of a couple of images
}
\end{lstlisting}

What is the problem?  (Hint: read both signatures out loud)

\end{frame}



%============================================================
\subsection{\mln Programming Paradigm}


%........................................................................
\begin{frame}[fragile]
  \frametitle{``Classical'' Object-Orientation}

In ``classical'' OO programming (OOP), we would write:
\begin{lstlisting}
Dpoint operator+(const Dpoint& dp1, const Dpoint& dp2) {
  ... // addition of a couple of delta-points
}

Image operator+(const Image& ima1, const Image& ima2) {
  ... // addition of a couple of images
}
\end{lstlisting}
which is clearly not ambiguous (but slow at run-time...)

where \code{Dpoint} and \code{Image} are abstract classes.

\end{frame}


%........................................................................
\begin{frame}[fragile]
  \frametitle{Between OOP and Genericity}

Can we try to mix OOP and Generic Programming (GP)?

\medskip

That is, getting something between:

\begin{lstlisting}
Dpoint operator+(const Dpoint& dp1, const Dpoint& dp2) {
  ...
}
\end{lstlisting}
and
\begin{lstlisting}
template <typename D>
D operator+(const D& dp1, const D& dp2) {
  ...
}
\end{lstlisting}

\end{frame}



%........................................................................
\begin{frame}[fragile]
  \frametitle{\mln Paradigm}

Yes:

\begin{lstlisting}
template <typename D>
D operator+(const Dpoint<D>& dp1, const Dpoint<D>& dp2) {
  ...
}
\end{lstlisting}

\begin{center}
  here \code{dp1} is a delta-point (\code{Dpoint}) of type \code{D}
\end{center}

\medskip

\begin{itemize}
\item it is not ambiguous at compile-time
\item it is efficient at run-time\\
  \scriptsize{}
\end{itemize}

\end{frame}


%........................................................................
\begin{frame}[fragile]
  \frametitle{Reading a \mln Signature of Routine}

\begin{lstlisting}
template <typename D>
D operator+(const Dpoint<D>& dp1, const Dpoint<D>& dp2) {
  ...
}
\end{lstlisting}

\medskip

\begin{center}
  The \code{operator+} takes a couple of \code{Dpoint} of type
  \code{D} and returns the same type.
\end{center}

\end{frame}



%============================================================
\subsection{How does it Work}


%........................................................................
\begin{frame}[fragile]
  \frametitle{Reading Again}

In \mln:
\begin{center}
  when we have something like ``\lstinline@Point<P>& p@'' \\
  it means that \code{p} is actually a point of type \code{P}
\end{center}

For instance, if the type of \code{p} is \code{point2d}, then
``another'' type for \code{p} is \lstinline@Point<point2d>@.

\smallskip

So
\begin{itemize}
\item in ``\lstinline@Point<P>& p@'', \code{P} is the \emph{exact} type of \code{p}
\item a type of point \code{P} derives from the abstraction \lstinline@Point<P>@
\end{itemize}

\end{frame}


%........................................................................
\begin{frame}[fragile]
  \frametitle{Conclusion}

More generally:

\begin{block}{Abstractions and Exact Types}
  When a concrete class \code{T} is related to an abstraction named \code{Abstraction},
  then \code{T} derives from \lstinline@Abstraction<T>@.

  \smallskip

  Every abstraction in \mln has exactly one parameter, which
  represents its exact type.
\end{block}

\end{frame}



%........................................................................
\begin{frame}[fragile]
  \frametitle{Examples}

  \begin{center}
  \begin{tabular}{lp{5mm}l}
    the class && derives from (``is a'') \\
    &&\\
    \lstinline@point2d@ && \lstinline@Point< point2d >@ \\
    \lstinline@point3d@ && \lstinline@Point< point3d >@ \\
    &&\\
    \lstinline@image2d<float>@ && \lstinline@Image< image2d<float> >@ \\
    \lstinline@image3d<int>@ && \lstinline@Image< image3d<int> >@ \\
    &&\\
    \lstinline@win::rectangle@ && \lstinline@Window< win::rectangle >@ \\
    &&\\
    \lstinline@box2d@ && \lstinline@Point_Set< box2d >@ \\
  \end{tabular}
  \end{center}

\end{frame}




%........................................................................
\begin{frame}[fragile]
  \frametitle{Excerpts from \mln}

\begin{lstlisting}
namespace level
{
  template <typename I, typename J>
  void paste(const Image<I>& data, Image<J>& destination);
}
\end{lstlisting}

\smallskip

\begin{lstlisting}
namespace morpho
{
  template <typename I, typename W>
  mln_concrete(I) erosion(const Image<I>& input, const Window<W>& win);
}
\end{lstlisting}

\smallskip

\begin{lstlisting}
namespace convert
{
  template <typename S>
  array_p<mln_point(S)> to_array_p(const Point_Set<S>& pset);
}
\end{lstlisting}

\end{frame}



%============================================================
\subsection{From Abstractions to Exact Types}


%........................................................................
\begin{frame}[fragile]
  \frametitle{A Hierarchy}

We have \code{dpoint2d}

\begin{lstlisting}
template <typename E>
class Dpoint
{};

class dpoint2d : public Dpoint< dpoint2d >
{
public:
  int& operator[](unsigned i);
  ...
};
\end{lstlisting}

\end{frame}



%........................................................................
\begin{frame}[fragile]
  \frametitle{A Troubleshooting}

This routine is almost (so \emph{not}) correct:
\begin{lstlisting}
template <typename D>
D operator+(const Dpoint<D>& dp1, const Dpoint<D>& dp2)
{
  D dp;
  for (unsigned i = 0; i < D::dim; ++i)
    dp[i] = dp1[i] + dp2[i];
    // above: \code{dp[i]} is OK
    // but \code{dp1[i]} and \code{dp2[i]} do \emph{not} compile!
  return dp;
}
\end{lstlisting}

because the \code{operator[]}
\begin{itemize}
\item is defined in concrete classes like \code{dpoint2d}
\item but \emph{not} in the abstract class
  \lstinline@Dpoint<dpoint2d>@
\end{itemize}

\end{frame}


%........................................................................
\begin{frame}[fragile]
  \frametitle{The \code{exact} Routine}

This updated routine works fine:

\begin{lstlisting}
template <typename D>
D operator+(const Dpoint<D>& dp1_, const Dpoint<D>& dp2_)
{
  const D& dp1 = exact(dp1_); // Cast to the exact type.
  const D& dp2 = exact(dp2_);
  D dp;
  for (unsigned i = 0; i < D::dim; ++i) dp[i] = dp1[i] + dp2[i];
  return dp;
}
\end{lstlisting}

\begin{block}{Exact}
  The ``\code{exact}'' routine allows getting a variable with the
  exact type of an object.
\end{block}

\end{frame}



%........................................................................
\begin{frame}[fragile]
  \frametitle{Remember about Genericity}

When this routine is called with \code{dpoint2d} objects, then
the compiled version is this one:

\begin{lstlisting}
dpoint2d operator+(const Dpoint<dpoint2d>& dp1_, const Dpoint<dpoint2d>& dp2_)
\end{lstlisting}

and its definition is finally:
\begin{lstlisting}
{
  dpoint2d dp;
  dp[0] = dp1[0] + dp2[0];
  dp[1] = dp1[1] + dp2[1];
  return dp;
}
\end{lstlisting}

\end{frame}



%........................................................................
\begin{frame}
  \frametitle{Conclusion}

This routine:
  \smallskip
  %

\begin{itemize}
\item is generic\\
  \scriptsize{it works for any delta-point type}
  \smallskip
  %
\item is fast,
  \scriptsize{you cannot get more efficient code}
  \smallskip
  %
\item is user-friendly,
  \scriptsize{just write ``\code{dp1 + dp2}'' to add a couple of delta-points}
\end{itemize}

\end{frame}



%........................................................................
\begin{frame}[fragile]
  \frametitle{Exercise 1}

Explain the code below:

\begin{lstlisting}
template <typename I>
void set(const Image<I>& ima_, const mln_point(I)& p, const mln_value(I)& v)
{
  const I& ima = exact(ima_);
  ima(p) = v;
}
\end{lstlisting}

\smallskip

What do have we in image classes?

\end{frame}


%........................................................................
\begin{frame}[fragile]
  \frametitle{Exercise 2}

Explain the code below:

\begin{lstlisting}
template <typename I, typename H>
void oper(Image<I>& f_, const Function_v2v<H>& h_)
{
  I& f = exact(f);
  const H& h = exact(h_);
  mln_piter(I) p(f.domain());
  for_all(p)
    f(p) = h(f(p));
}
\end{lstlisting}

\smallskip

What do have we in image classes?

\end{frame}



%############################################################
\end{document}

%%% Local Variables:
%%% mode: latex
%%% eval: (ispell-change-dictionary "american")
%%% TeX-master: t

% LocalWords: IP interoperability morphers Milena dpoint mln const typename dp
% LocalWords: int coord resp OOP GP OO namespace morpho pset
