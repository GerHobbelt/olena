\documentclass{beamer}

 \mode<article> % only for the article version
 {
   \usepackage{fullpage}
   \usepackage{hyperref}
 }


 \mode<presentation>
 {
   \setbeamertemplate{background canvas}[vertical shading][bottom=red!10,top=blue!10]
   \usetheme{Warsaw}
   \usefonttheme[onlysmall]{structurebold}
 }

%\setbeamercolor{math text}{fg=green!50!black}
%\setbeamercolor{normal text in math text}{parent=math text}

\usepackage{numberslides}

\usepackage{pgf,pgfarrows,pgfnodes,pgfautomata,pgfheaps,pgfshade}
\usepackage{amsmath,amssymb}
\usepackage[english]{babel}
\usepackage[latin1]{inputenc}
\usepackage{colortbl}

%\usepackage{lmodern}
%\usepackage[T1]{fontenc} 
%\usepackage{times}

\usepackage{multicol}

\usepackage{graphicx}
\graphicspath{{figures/}}

\usepackage{xspace}
\usepackage{ulem}
\usepackage{url}

\usepackage{listings}
\lstloadlanguages{[ISO]C++}
\lstset{language=[ISO]C++,
  texcl=true,
  columns=fullflexible,
  basicstyle={\scriptsize\sffamily}, % normal footnote small scriptsize tiny
  commentstyle=\itshape,
  showstringspaces=false,
  numberstyle=\tiny,
  morekeywords={where, auto, concept, concept_map, axiom, late_check},
  morecomment=[s]{/*}{*/}
}

%-------------------------------
% newalg

\usepackage{newalg}

% layout
\newcommand{\tab}{\hspace*{5mm}}
\newcommand{\comment}[1]{\texttt{// #1}}
\newcommand{\change}[1]{\comment{change:~#1}}
\newcommand{\invariant}{\textbf{invariant}~}
\newcommand{\invariants}{\textbf{invariants:}~}
  
% macros
\newcommand{\escapemath}[1]{\ensuremath{\mbox{#1}}}
\newcommand{\mktype}[1]{\escapemath{\textbf{\textit{#1}}}}
\newcommand{\mkkw}[1]{\escapemath{\textbf{~#1~}}}

% types
\newcommand{\tbool}{\mktype{bool}}
\newcommand{\tint}{\mktype{integer}}
\newcommand{\tunsigned}{\mktype{unsigned}}

% kw
\newcommand{\OR}{\mkkw{or}}
\newcommand{\AND}{\mkkw{and}}
\newcommand{\NOT}{\mkkw{not}}
\newcommand{\TO}{\mkkw{to}}

% colors
\definecolor{darkyellow}{rgb}{0.5,0.5,0.1}
\definecolor{darkgreen}{rgb}{0.1,0.7,0.1}
\definecolor{mediumgreen}{rgb}{0.6,0.3,0.1}
\definecolor{mediumblue}{rgb}{0.2,0.4,0.6}
\definecolor{mediumpink}{rgb}{0.6,0.1,0.6}
\definecolor{darkpink}{rgb}{0.3,0.0,0.3}
\definecolor{darkgray}{rgb}{0.4,0.4,0.4}

% colored tokens
\newcommand{\blueIterator}{\color{blue}{Iterator}}
\newcommand{\blueArrayIterator}{\color{blue}{array\_iterator}}
\newcommand{\greenI}{\color{darkgreen}{I}}

% other tokens
\newcommand{\bfarrayiteratorT}{\textbf{array\_iterator$<$T$>$}\xspace}
\newcommand{\bfhelloiteratorI}{\textbf{hello\_iterator$<$I$>$}\xspace}
\newcommand{\bfhelloiterator}{\textbf{hello\_iterator}\xspace}
\newcommand{\bfForwardIterator}{\textit{\textbf{Forward\_Iterator}}\xspace}
\newcommand{\bfWriteOnlyIterator}{\textit{\textbf{Write\_Only\_Iterator}}\xspace}
\newcommand{\bfBackwardIterator}{\textit{\textbf{Backward\_Iterator}}\xspace}
\newcommand{\bfIterator}{\textit{\textbf{Iterator}}\xspace}
\newcommand{\bfinit}{\textbf{init}\xspace}
\newcommand{\bfisvalid}{\textbf{is\_valid}\xspace}
\newcommand{\bfnext}{\textbf{next}\xspace}
\newcommand{\bfprev}{\textbf{prev}\xspace}
\newcommand{\bfI}{\textbf{I}\xspace}
\newcommand{\bfreverseI}{\textbf{reverse$<$I$>$}\xspace}
\newcommand{\bfreverseT}{\textbf{reverse$<$T$>$}\xspace}
\newcommand{\bfreversearrayiteratorT}{\textbf{reverse$<$~\bfarrayiteratorT~$>$}\xspace}
\newcommand{\bfreverse}{\textbf{reverse}\xspace}
\newcommand{\bffT}{\textbf{f$<$T$>$}\xspace}
\newcommand{\bff}{\textbf{f}\xspace}
\newcommand{\bfT}{\textbf{T}\xspace}
\newcommand{\bftop}{\textbf{top}\xspace}



% end of newalg
%-------------------------------


\newcommand{\bplus}{\textbf{+}\xspace}
\newcommand{\bminus}{\textbf{\LARGE{-}}\xspace}


\newcommand{\scoop}{\textsc{scoop}\xspace}



\setbeamercovered{dynamic}


\title[Static Programming in C++]{Static Programming in C++}

\author[T. G\'eraud and R. Levillain]{Thierry G\'eraud and Roland Levillain}

\institute[LRDE]{EPITA Research and Development Laboratory (LRDE)}

\date[LRDE February 2006]{LRDE February 2006}

\subject{}

%\pgfdeclaremask{lrde}{lrde-logo-mask}
%\pgfdeclareimage[mask=lrde,width=0.6cm]{lrde-logo}{lrde}
%\pgfdeclareimage[width=0.6cm]{lrde-logo}{lrde}

%\logo{\vbox{\hbox to 1cm{\hfil\pgfuseimage{lrde-logo}}}}
\logo{}




%\includeonly{}




\begin{document}

\frame{\titlepage}

\section<presentation>*{Outline}

\begin{frame}
  \frametitle{Outline}
{\scriptsize
  \tableofcontents[part=1,pausesections]
}
\end{frame}

\AtBeginSubsection[]
{
  \begin{frame}<beamer>
    \frametitle{Outline}
{\scriptsize
    \tableofcontents[current,currentsubsection]
}
  \end{frame}
}

\part<presentation>{Main Talk}



%============================================================
\section*{Introduction}

%........................................................................
\begin{frame}
  \frametitle{Context of our work}

FIXME: olena...

\end{frame}




%============================================================

%============================================================
\section{A Tour of Several Paradigms}

%........................................................................
\begin{frame}
  \frametitle{Intents}

  FIXME: context + get the best of both OOP and GP.

\end{frame}



%------------------------------------------------------------
\subsection{Classical OOP}


%........................................................................
\begin{frame}
  \frametitle{Classical Object-Oriented Programming}

  \begin{itemize}
    % 
  \item we design class hierarchies
    \begin{itemize}
    \item class inheritance translates the \textit{is-a} relationship
    \item a concrete class usually derives from an abstract class
    \end{itemize}
    \smallskip
    % 
  \item an abstract class
    \begin{itemize}
    \item maps an \textit{abstraction}
    \item describes an \textit{interface}
    \end{itemize}
    \smallskip
    % 
  \item a concrete class
    \begin{itemize}
    \item is a realization of its immediate abstraction
    \item implements its interface and those inherited
    \end{itemize}
    % 
  \end{itemize}

\end{frame}



%........................................................................
\begin{frame}
  \frametitle{Simple Class Hierarchy + Algorithm}

\begin{multicols}{2}
%
  \begin{center}
    \includegraphics[scale=.7]{simpleoo}
  \end{center}
%
\columnbreak
%
{\scriptsize
\begin{algorithm}{count}{it : Iterator}
  counter : \tunsigned \= 0 \\
  it.init() \\
  \begin{WHILE}{it.is\_valid()}
    \tab counter \= counter + 1 \\
    \tab it.next()
  \end{WHILE} \\
  \RETURN counter
\end{algorithm}
}
%
\end{multicols}

\end{frame}



%........................................................................
\begin{frame}[fragile]
  \frametitle{Translation into C++ OOP}


\begin{multicols}{2}
%
\begin{lstlisting}[escapechar=@]
class @\blueIterator@
{
public:
  virtual void init() = 0;
  virtual void next()  = 0;
  virtual bool is_valid() const = 0;
  virtual ~Iterator() {}
};

template <typename T>
class @\color{mediumblue}{array\_iterator}@ : public @\blueIterator@
{
public:
  virtual void init() { ... }
  virtual void next() { ... }
  virtual bool is_valid() const { ... }
  ...
};
\end{lstlisting}
%
\columnbreak
%
\begin{lstlisting}[escapechar=@]
unsigned count(@\blueIterator@& it)
{
  unsigned counter = 0;
  it.init();
  while (it.is_valid()) {
    counter = counter + 1;
    it.next();
  }
  return counter;
}
\end{lstlisting}
%
\end{multicols}

\end{frame}



%........................................................................
\begin{frame}
  \frametitle{Conclusion About C++ OOP}


\begin{block}{C++ OOP in a Nutshell}
  \begin{itemize}
  \item[\bplus] Abstract classes \\
    ~~~ {\tiny{$\leadsto$ ~ abstractions clearly appear in programs and allow for overloading of algorithms}}
  \item[\bplus] The OO Polymorphism \\
    ~~~ {\tiny{$\leadsto$ ~ method dispatch is a very convenient and straightforward tool}}
  \item[\bminus] Poor efficiency \\
    ~~~ {\tiny{$\leadsto$ ~ ``virtual'' methods are penalizing when involved in intensive computation}}
  \item[\bminus] No support for covariance and virtual types \\
    ~~~ {\tiny{$\leadsto$ ~ see later}}
  \end{itemize}
\end{block}

\end{frame}




%------------------------------------------------------------
\subsection{Classical Generic Programming (GP)}


%........................................................................
\begin{frame}[fragile]
  \frametitle{Documentation and Implementation}

\begin{multicols}{2}
%
  \begin{center}
    \includegraphics[scale=.7]{simplegp}
  \end{center}
%
\columnbreak
%
{\scriptsize
\begin{itemize}
\item in code:
  \begin{itemize}
  \item no actual abstract classes
  \item only implementation classes
  \end{itemize}
  \smallskip
\item limited OOP:
  \begin{itemize}
  \item abstractions only appear in documentation
  \item inheritance is just a factoring tool
  \end{itemize}
\end{itemize}
}
%
\end{multicols}

\end{frame}



%........................................................................
\begin{frame}[fragile]
  \frametitle{Translation into C++ Generic Programming}

\begin{multicols}{2}
%
\begin{lstlisting}[escapechar=@]
/* documentation:
 * 
 * Iterator
 * {
 *   void init();
 *   void next();
 *   bool is_valid() const;
 * };
 */

template <typename T>
class array_iterator
{
public:
  void init() { ... }
  void next() { ... }
  bool is_valid() const { ... }
  ...
};
\end{lstlisting}
%
\columnbreak
%
\begin{lstlisting}[escapechar=@]
template <typename @\greenI@>
unsigned count(@\greenI@& it)
{
  unsigned counter = 0;
  it.init();
  while (it.is_valid()) {
    counter = counter + 1;
    it.next();
  }
  return counter;
}
\end{lstlisting}
%
\end{multicols}

\end{frame}



%........................................................................
\begin{frame}
  \frametitle{Conclusion About C++ GP}

\begin{block}{C++ GP in a Nutshell}
  \begin{itemize}
  \item[\bplus]  Efficiency \\
    ~~~ {\tiny{$\leadsto$ ~ algorithms do not pay the cost of abstractions}}
  \item[\bminus] Implicit abstractions \\
    ~~~ {\tiny{$\leadsto$ ~ abstractions are not mapped into code so program expressiveness is limited}}
  \item[\bminus] No support for constrained genericity \\
    ~~~ {\tiny{$\leadsto$ ~ being able to overload generic algorithms is not trivial}}
  \item[\bminus] Factoring code using inheritance is usually hard \\
    ~~~ {\tiny{$\leadsto$ ~ postponing definitions to sub-classes and
        overriding do not work well without the 'virtual'
        keyword}}
  \end{itemize}
\end{block}

\end{frame}




%------------------------------------------------------------
\subsection{Generic Programming Evolution with C++0x}



%........................................................................
\begin{frame}[fragile]
  \frametitle{Concepts and Implementation}

\begin{multicols}{2}
%
  \begin{center}
    \includegraphics[scale=.7]{simple0x}
  \end{center}
%
\columnbreak
%
{\scriptsize
\begin{itemize}
\item in code:
  \begin{itemize}
  \item concepts + implementation classes
  \item no need of setting explicit relations between them
    % concepts and classes
  \end{itemize}
  \smallskip
\item consequences:
  \begin{itemize}
  \item stronger type-checking than in GP
  \item great decoupling between concepts and classes
  \item inheritance remains a factoring tool
  \end{itemize}
\end{itemize}
}
%
\end{multicols}

\end{frame}



%........................................................................
\begin{frame}[fragile]
  \frametitle{Translation into C++0x}

\begin{multicols}{2}
%
\begin{lstlisting}[escapechar=@]
auto concept @\blueIterator@ <typename @\greenI@>
{
  void I::init();
  void I::next();
  bool I::is_valid() const;
};

template <typename T>
class array_iterator
{
public:
  void init() { ... }
  void next() { ... }
  bool is_valid() const { ... }
  ...
};
\end{lstlisting}
%
\columnbreak
%
\begin{lstlisting}[escapechar=@]
template <typename @\greenI@>
where @\blueIterator@<@\greenI@>
unsigned count(@\greenI@& it)
{
  unsigned counter = 0;
  it.init();
  while (it.is_valid()) {
    counter = counter + 1;
    it.next();
  }
  return counter;
}
\end{lstlisting}
%
\end{multicols}

\end{frame}




%........................................................................
\begin{frame}
  \frametitle{Conclusion About C++0x}

\begin{block}{C++0x in a Nutshell}
  \begin{itemize}
  \item[\bplus] Efficiency, expressiveness, and type safety \\
    ~~~ {\tiny{$\leadsto$ ~ most drawbacks of OOP and GPP disappear!}}
  \item[\bminus] Not really OO \\
    ~~~ {\tiny{$\leadsto$ ~ like with GP we cannot take advantage of many designs related to ``classical'' inheritance}}
  \item[\bminus] Not yet part of the C++ standard \\
    ~~~ {\tiny{$\leadsto$ ~ so is there something to do now but waiting?}}
  \end{itemize}
\end{block}

\end{frame}




%------------------------------------------------------------
\subsection{Mixing OOP and GP}



%........................................................................
\begin{frame}
  \frametitle{Combining Inheritance and Genericity}

\begin{multicols}{2}
%
  \begin{center}
    \includegraphics[scale=.7]{simplescoop1}
  \end{center}
%
\columnbreak
%
\vspace*{.2\textheight}
\begin{itemize}
\item close to OOP
  \begin{itemize}
  \item we have class hierarchies
  \item but method lookup is solved at compile-time!
  \end{itemize}
\item close to GP
  \begin{itemize}
  \item algorithms are fast
  \item but their signatures are as strong as in C++0x!
  \end{itemize}
\end{itemize}
%
\end{multicols}

\end{frame}



%........................................................................
\begin{frame}[fragile]
  \frametitle{Translation into \scoop 1}

\begin{multicols}{2}
%
\begin{lstlisting}[escapechar=@]
template <typename @\greenI@>
class @\blueIterator@ : public @\color{darkgray}{Any}@<I>
{
public:
  void init() { this->@\color{darkgray}{exact()}@.impl_init(); }
  void next() { /* likewise */ }
  bool is_valid() const  { /* likewise */ }
};



template <typename T>
class @\blueArrayIterator@ : public @\blueIterator@< array_iterator<T> >
{
public:
  void impl_init() { ... }
  void impl_next() { ... }
  bool impl_is_valid() const { ... }
  ...
};
\end{lstlisting}
%
\columnbreak
%
\begin{lstlisting}[escapechar=@]
template <typename @\greenI@>
unsigned count(@\blueIterator@<@\greenI@>& it)
{
  unsigned counter = 0;
  it.init();
  while (it.is_valid()) {
    counter = counter + 1;
    it.next();
  }
  return counter;
}
\end{lstlisting}
%
\end{multicols}

\end{frame}




%........................................................................
\begin{frame}
  \frametitle{Conclusion About \scoop~1}

\begin{block}{\scoop~1 in a Nutshell}
  \begin{itemize}
  \item[\bplus] Efficiency, expressiveness, and type safety \\
    ~~~ {\tiny{$\leadsto$ ~ most drawbacks of OOP and GPP disappear!}}
  \item[\bplus] Really OO \\
    ~~~ {\tiny{$\leadsto$ ~ a mix between OOP and GP}}
  \item[\bminus] Not so trivial C++ \\
    ~~~ {\tiny{$\leadsto$ ~ yet not so difficult neither!}}
  \item[\bminus] Explicit inheritance \\
    ~~~ {\tiny{$\leadsto$ ~ just like in OOP...}}
  \end{itemize}
\end{block}

\end{frame}




%------------------------------------------------------------
\subsection{Recap + Comparison}



%........................................................................
\begin{frame}[fragile]
  \frametitle{Abstraction and Algorithm}

\hspace*{-3mm}
\begin{tabular}{|c|p{.45\textwidth}|p{.4\textwidth}|}
%
\hline
%
{\small OOP}
&
\begin{minipage}{.9\textwidth}
\begin{lstlisting}[escapechar=@]
class @\blueIterator@
{ ... };
\end{lstlisting}
\end{minipage}
&
\begin{minipage}{.9\textwidth}
\begin{lstlisting}[escapechar=@]
unsigned count(@\blueIterator@& it);
\end{lstlisting}
\end{minipage}
\\
%
\hline
%
{\small GP}
&
\begin{minipage}{.9\textwidth}
\begin{lstlisting}[escapechar=@]
/* documentation: Iterator
    ... */
\end{lstlisting}
\end{minipage}
&
\begin{minipage}{.9\textwidth}
\begin{lstlisting}[escapechar=@]
template <typename @\greenI@>
unsigned count(@\greenI@& it);
\end{lstlisting}
\end{minipage}
\\
%
\hline
%
{\small C++0x}
&
\begin{minipage}{.9\textwidth}
\begin{lstlisting}[escapechar=@]
auto concept @\blueIterator@ <typename @\greenI@>
{ ... };
\end{lstlisting}
\end{minipage}
&
\begin{minipage}{.9\textwidth}
\begin{lstlisting}[escapechar=@]
template <typename @\greenI@>
where @\blueIterator@<@\greenI@>
unsigned count(@\greenI@& it);
\end{lstlisting}
\end{minipage}
% template <@\blueIterator@ @\greenI@>
\\
%
\hline
%
{\small \scoop}
&
\begin{minipage}{.9\textwidth}
\begin{lstlisting}[escapechar=@]
template <typename @\greenI@>
class @\blueIterator@ : public @\color{darkgray}{Any}@<I>
{ ... }
\end{lstlisting}
\end{minipage}
&
\begin{minipage}{.9\textwidth}
\begin{lstlisting}[escapechar=@]
template <typename @\greenI@>
unsigned count(@\blueIterator@<@\greenI@>& it);
\end{lstlisting}
\end{minipage}
\\
%
\hline
%
\end{tabular}


\end{frame}



%........................................................................
\begin{frame}
  \frametitle{Paradigms At a Glance}

\newcommand{\badweak}{\color{red}{weak}}
\newcommand{\badexplicit}{\color{red}{explicit}}
\newcommand{\badno}{\color{red}{no}}
\newcommand{\badpoor}{\color{red}{poor}}

\newcommand{\yestrong}{\color{darkgreen}{strong}}
\newcommand{\yeimplicit}{\color{darkgreen}{implicit}}
\newcommand{\yeyes}{\color{darkgreen}{yes}}
\newcommand{\yegreat}{\color{darkgreen}{great}}

\newcommand{\ftypetotype}{{\scriptsize{$f: \mathit{type} \rightarrow \mathit{type}$}}}

\hspace*{-5mm}
\begin{tabular}{|c||c|c|c|c|c|}
\hline
            & OOP               & GP                    & C++0x               & \scoop            & \scoop 2            \\
\hline
\hline
2003 C++    & \yeyes            & \yeyes                & \badno               & \yeyes            & \yeyes             \\
inheritance & ok                & \badpoor              & \badpoor             & \yegreat          & \yegreat           \\
abstraction & {\bf class}       & \textsc{rtfm}         & {\bf concept}        & {\bf class}       & {\bf struct}       \\
relation    & \badexplicit      & \yeimplicit           & \yeimplicit          & \badexplicit      & quasi-\yeimplicit  \\
signature   & \yestrong         & \badweak              & \yestrong            & \yestrong         & \yestrong          \\
efficiency  & \badpoor          & \yegreat              & \yegreat             & \yegreat          & \yegreat           \\
\hline
\hline
\ftypetotype& \badno            & \badno                & \badno               & \badno            & \yeyes             \\
\hline
\end{tabular}

\end{frame}


%%% Local Variables:
%%% mode: latex
%%% eval: (ispell-change-dictionary "american")
%%% TeX-master: "slides"
%%% End:



%============================================================

%============================================================
\section{What Do We Want?}

%------------------------------------------------------------
\subsection{Expectations}


%........................................................................
\begin{frame}
  \frametitle{What Would Be Great? (\scoop~1)}

Expectations are:
\begin{itemize}
\item factor code through class hierarchies
  \begin{itemize}
  \item either with ``classical'' hierarchies\\
    ~ {\tiny{$\leadsto$ ~ abstract classes on top of implementation classes}}
  \item or with ``modern'' generic programming\\
    ~ {\tiny{$\leadsto$ ~ hierarchies of concepts \emph{and} hierarchies of implementation classes}}
  \end{itemize}
\item get virtual types in C++
\end{itemize}

\smallskip

with:
\begin{itemize}
\item strong type checking
\item method and type lookup performed at compile-time
\end{itemize}

\smallskip

and...

\end{frame}



%........................................................................
\begin{frame}
  \frametitle{What Would Be Great? (\scoop~2)}

and...

\smallskip

\begin{itemize}
\item separate concepts and implementation classes
%
\smallskip
%
\item have types:
  \begin{itemize}
  \item that are defined over other types
  \item whose definition is given once
  \item that modify those types
  \item and that are able to keep their specificities
  \end{itemize}
\end{itemize}

\end{frame}




%........................................................................
\begin{frame}
  \frametitle{The Need for Virtual Types}


\begin{multicols}{2}
%
  \begin{center}
    \includegraphics[scale=.7]{simpleoovt}
  \end{center}
%
\columnbreak
%
\begin{itemize}
\item we sometimes need covariant methods
\item C++ is limited to invariance...
\item type safety cannot be guarantied at compile-time
\end{itemize}
%
\end{multicols}

\end{frame}



%........................................................................
\begin{frame}[fragile]
  \frametitle{Translation into Valid C++}

\begin{multicols}{2}
%
\begin{lstlisting}[escapechar=@]
class @\blueIterator@
{
public:
  virtual void set_value(void* v) = 0;
  ...
};

template <typename T>
class @\blueArrayIterator@ : public @\blueIterator@
{
public:
  typedef T value;
  virtual void set_value(void* v)
  {
    this->v_ = *(value*)v;
  }
  ...
};
\end{lstlisting}
%
\columnbreak
%
\begin{lstlisting}[escapechar=@]
void fill(@\blueIterator@& it, @\color{darkyellow}{void*}@ v)
{
  it.init();
  while (it.is_valid())
    it.set_value(v);
}
\end{lstlisting}
%
\end{multicols}

\end{frame}




%........................................................................
\begin{frame}
  \frametitle{The Need for Virtual Types (2/2)}

\begin{multicols}{2}
%
  \begin{center}
    \includegraphics[scale=.7]{simplevt}
  \end{center}
%
\columnbreak
%
\begin{itemize}
\item solution:
  \begin{itemize}
  \item virtual types
  \item behaving just like virtual methods
  \item expressing that 'value' is an associated type
  \end{itemize}
\smallskip
\item but:
  \begin{itemize}
  \item C++ does not feature virtual types
  \item we also expect them to be statically type checked
  \end{itemize}
\end{itemize}
%
\end{multicols}

\end{frame}




%........................................................................
\begin{frame}[fragile]
  \frametitle{Translation into Pseudo-C++}

\begin{multicols}{2}
%
\begin{lstlisting}[escapechar=@]
class @\blueIterator@
{
public:
  virtual typedef @\color{darkyellow}{value}@ = 0;
  virtual void set_value(value v) = 0;
  ...
};

template <typename T>
class @\blueArrayIterator@ : public @\blueIterator@
{
public:
  virtual typedef T value;
  virtual void set_value(value v)
  {
    this->v_ = v;
  }
  ...
};
\end{lstlisting}
%
\columnbreak
%
\begin{lstlisting}[escapechar=@]
@\color{red}{/* this slide is NOT valid C++ */}@

void fill(@\blueIterator@& it, @\blueIterator@::@\color{darkyellow}{value}@ v)
{
  it.init();
  while (it.is_valid())
    it.set_value(v);
}
\end{lstlisting}
%
\end{multicols}

\end{frame}



%------------------------------------------------------------
\subsection{Static OO Hierarchy with Virtual Types (\scoop~1)}



%........................................................................
\begin{frame}
  \frametitle{Migration towards \scoop: Static Hierarchy}

\vspace*{-7mm}
  \begin{center}
    \includegraphics[scale=.55]{vt1}
  \end{center}

\end{frame}




%........................................................................
\begin{frame}
  \frametitle{\scoop Static Hierarchy (1/3)}

  \begin{block}{Rule}
    An abstract class takes exactly one parameter: \textbf{Exact}.
  \end{block}

  \medskip

  \begin{itemize}
    % 
    \item this parameter holds its exact type\\
      ~~~ {\scriptsize $\leadsto$ ~ e.g., when an object is a \textbf{rabbit}, it is an \textbf{Animal$<$rabbit$>$} }
      % 
      \smallskip
      % 
    \item an abstract class does not need any extra parameter\\
      ~~~ {\scriptsize $\leadsto$ ~ understand that everything can be known/fetched from \textbf{Exact}! }\\
      ~~~ {\scriptsize $\leadsto$ ~ \textbf{Iterator} does NOT need \textbf{T} as a parameter }\\
      ~~~ {\scriptsize $\leadsto$ ~ and we do not want to guess what else \textbf{Iterator} could need }
      % 
      \smallskip
      % 
    \item a concrete class sets the parameter of its abstract super class
    \end{itemize}

\end{frame}





%........................................................................
\begin{frame}
  \frametitle{\scoop Static Hierarchy (2/3)}


  \begin{block}{Rule}
    Methods in \scoop slightly differ from their classical writing.
  \end{block}

  \medskip

  \begin{itemize}
  \item an abstract method is now \textit{pseudo}-abstract\\
    ~~~ {\scriptsize $\leadsto$ ~ there is no ``\textbf{virtual ... = 0}'' }\\
    ~~~ {\scriptsize $\leadsto$ ~ there is some code: }
    % 
    \smallskip
    % 
  \item the dispatch mechanism is hand-written
    % 
    \smallskip
    % 
  \item the names of actual concrete methods are prefixed by \textbf{impl\_}\\
    ~~~ {\scriptsize $\leadsto$ ~ first for disambiguation purpose } \\
    ~~~ {\scriptsize $\leadsto$ ~ second to allow ``class extension from the top'' (see later) }
  \end{itemize}

\end{frame}



%........................................................................
\begin{frame}
  \frametitle{\scoop Static Hierarchy (3/3)}

  \begin{block}{Rule}
    The upper most abstract class derives from \textbf{Any}.
  \end{block}

  \medskip

  \begin{itemize}
    % 
  \item so some equipment is inherited\\
    ~~~ {\scriptsize $\leadsto$ ~ for instance the \textbf{exact()} downcast method }
    % 
    \smallskip
    % 
  \item {\color{red}{warning:}}
    \begin{itemize}
    \item this is only ``\scoop~1'' stuff
    \item i.e., turn a ``classical'' OO hierarchy into the static world
    \item we will see more elaborate design later...
    \end{itemize}
    % 
  \end{itemize}

\end{frame}





%........................................................................
\begin{frame}[fragile]
  \frametitle{\scoop Virtual Type (1/3)}

  First realize that we want something like:

\medskip

\hspace*{0.1\textwidth}
\begin{minipage}{0.8\textwidth}
\begin{lstlisting}[basicstyle={\tiny\sffamily}]
template <typename Exact>
struct Iterator
{
  typedef typename Exact::value value;
  /*
   * so 'value' can be used in:
   * void set_value(value v) { ... }
   */
};

template <typename T>
struct array_iterator : Iterator< array_iterator<T> >
{
  typedef T value;
};
\end{lstlisting}
\end{minipage}

\medskip

except that this code does not compile...\\
{\small (both classes are in mutual recursion at compile-time)}
\end{frame}




%........................................................................
\begin{frame}
  \frametitle{Migration towards \scoop (continued)}

%\vspace*{-5mm}
\hspace*{-5mm}
\includegraphics[scale=.55]{vt2}

\end{frame}





%........................................................................
\begin{frame}
  \frametitle{\scoop Virtual Type (2/3)}

  \begin{block}{Rule}
    Split virtual type declarations/definitions from classes.
  \end{block}

  \medskip

  \begin{itemize}
    % 
  \item the \textbf{vtype} structure is specialized for each client class\\
    ~~~ {\scriptsize $\leadsto$ ~ it holds declarations and definitions of virtual types }\\
    ~~~ {\scriptsize $\leadsto$ ~ a declaration is performed with \textbf{stc::abstract} }\\
    ~~~ {\scriptsize $\leadsto$ ~ definitions can be overridden through inheritance }
    % 
    \smallskip
    % 
  \item client classes have to ``import'' virtual types\\
    ~~~ {\scriptsize $\leadsto$ ~ with \textbf{stc\_typename} when a virtual type is declared {\tiny(or defined for the 1st time)} }\\
    ~~~ {\scriptsize $\leadsto$ ~ with \textbf{stc\_using} to use a virtual type in a sub-class {\tiny(when already in a super one)} }
    % 
  \end{itemize}

\end{frame}




%........................................................................
\begin{frame}[fragile]
  \frametitle{\scoop Virtual Type (3/3)}

  \begin{block}{Rule}
    From a client point of view, associated types are already solved.
  \end{block}

  \medskip

  \begin{itemize}
    % 
    \item use the classical access syntax \\
      ~~~ {\scriptsize $\leadsto$ ~ either \textbf{Iterator$<$I$>$::value} or \textbf{I::value} }
      % 
      \smallskip
      % 
    \item to make sure a definition is not overridden use \textbf{stc::final$<$..$>$} \\
      ~~~ {\scriptsize $\leadsto$ ~ Cf. the upcoming sample code }
      % 
      \smallskip
      % 
    \item ...\\
      ~~~ {\scriptsize $\leadsto$ ~ actually we do not really have ``actual'' virtual types (but who cares?) }
    \end{itemize}

\end{frame}




%........................................................................
\begin{frame}[fragile]
  \frametitle{Nothing But Code (1/3)}

%
\begin{lstlisting}[escapechar=@,basicstyle={\tiny\sffamily}]
#include <stc/scoop.hh>

namespace abc
{
  @\color{mediumblue}{stc\_equip\_namespace}@;
  @\color{mediumblue}{stc\_decl\_associated\_type}@(value);


# define templ@~~~~~~~~@template <typename Exact>
# define classname@~~@Iterator
# define @\color{mediumpink}{current}~~~~~~\color{red}{Iterator$<$Exact$>$}@
# define @\color{mediumpink}{super}~~~~~~~~@stc::any<Exact>

  @\color{mediumblue}{stc\_Header}@;
    typedef stc::abstract value;
  @\color{mediumblue}{stc\_End}@;

  template <typename Exact>
  struct Iterator : public super
  {
    @\color{mediumblue}{stc\_typename}@(value);
    void next()@~~~~~~~~~~~~~~~~~~~@{ this->exact().impl_next();  }
    bool is_valid() const@~~~~~~~~@{ return this->exact().impl_is_valid(); }
    void set(const value& v)@~~@{ this->exact().impl_set(v);  }
  };

} // end of namespace abc
\end{lstlisting}
%

\end{frame}




%........................................................................
\begin{frame}[fragile]
  \frametitle{Nothing But Code (1/3) --- Desugared Version}

%
\begin{lstlisting}[escapechar=@,basicstyle={\tiny\sffamily}]
#include <stc/scoop.hh>

namespace abc
{
  @\color{mediumblue}{stc\_equip\_namespace}@;
  @\color{mediumblue}{stc\_decl\_associated\_type}@(value);


  template <typename Exact> struct Iterator;  // fwd decl

  template <typename Exact> @\hspace*{4cm} \color{mediumgreen}{SUGAR-FREE VERSION!}@
  struct vtypes< @\color{red}{Iterator$<$Exact$>$}@ >
  {
    typedef stc::any<Exact> @\color{mediumpink}{super\_type}@;
    typedef stc::abstract value;
  };
  
  template <typename Exact>
  struct Iterator : public stc::any<Exact>
  {
    typedef @\color{mediumblue}{stc\_type}@(Exact, value) value;
    void next()@~~~~~~~~~~~~~~~~~~~@{ this->exact().impl_next();  }
    bool is_valid() const@~~~~~~~~@{ return this->exact().impl_is_valid(); }
    void set(const value& v)@~~@{ this->exact().impl_set(v);  }
  };

} // end of namespace abc
\end{lstlisting}
%

\end{frame}





%........................................................................
\begin{frame}[fragile]
  \frametitle{Nothing But Code (2/3)}

%
\begin{lstlisting}[escapechar=@,basicstyle={\tiny\sffamily}]
namespace abc
{
# define templ@~~~~~~~~@template <typename T>
# define classname@~~@array_iterator
# define @\color{mediumpink}{current}~~~~~~\color{red}{array\_iterator$<$T$>$}@
# define @\color{mediumpink}{super}~~~~~~~~@Iterator< current >

  @\color{mediumblue}{stc\_Header}@;
    typedef stc::final<T> value;
  @\color{mediumblue}{stc\_End}@;

  template <typename T>
  class array_iterator : public super
  {
  public:
    @\color{mediumblue}{stc\_using}@(value);
    void impl_next()@~~~~~~~~~~~~~~~~~~~@{ i_ = i_ + 1; }
    bool impl_is_valid() const@~~~~~~~~@{ return i_ >= 0 and i_ < n_; }
    void impl_set(const value& v)@~~@{ v_ = v; }
    array_iterator(int n)@~~~~~~~~~~~~~~~@: i_(0), n_(n) {}
  protected:
    int   i_, n_;
    value v_;
  };

} // end of namespace abc
\end{lstlisting}
%

\end{frame}



%........................................................................
\begin{frame}[fragile]
  \frametitle{Nothing But Code (2/3) --- Desugared Version}

%
\begin{lstlisting}[escapechar=@,basicstyle={\tiny\sffamily}]
namespace abc
{
  template <typename T> array_iterator;  // fwd decl

  template <typename T>
  struct vtypes< @\color{red}{array\_iterator$<$T$>$}@ >
  {
    typedef Iterator< array_iterator<T> > @\color{mediumpink}{super\_type}@;
    typedef stc::final<T> value;
  }; @\hspace*{6.7cm} \color{mediumgreen}{SUGAR-FREE VERSION!}@
  
  template <typename T>
  class array_iterator : public Iterator< array_iterator<T> >
  {
  public:
    typedef typename Iterator< array_iterator<T> >::value value;
    void impl_next()@~~~~~~~~~~~~~~~~~~~@{ i_ = i_ + 1; }
    bool impl_is_valid() const@~~~~~~~~@{ return i_ >= 0 and i_ < n_; }
    void impl_set(const value& v)@~~@{ v_ = v; }
    array_iterator(int n)@~~~~~~~~~~~~~~~@: i_(0), n_(n) {}
  protected:
    int   i_, n_;
    value v_;
  };

} // end of namespace abc
\end{lstlisting}
%

\end{frame}


%........................................................................
\begin{frame}[fragile]
  \frametitle{Nothing But Code (3/3)}

%
\begin{lstlisting}[escapechar=@,basicstyle={\tiny\sffamily}]
namespace abc
{
  // \color{red}{algorithm}

  template <typename @\greenI@>
  void fill(@\blueIterator@<@\greenI@>& it,@~~@typename @\greenI@::@\color{darkyellow}{value}@ v)
  {
    it.init();
    while (it.is_valid())
      it.set_value(v);
  }

} // end of namespace abc


// \color{red}{main}

int main()
{
  abc::array_iterator<int> i(7);
  abc::algo(i, 51);
}

\end{lstlisting}
%

\end{frame}



%........................................................................
\begin{frame}[fragile]
  \frametitle{Going Further with \scoop 1}

%
  {\small A static {C++} object-oriented programming ({SCOOP})
    paradigm mixing benefits of traditional {OOP} and generic
    programming.
  }
%
  {\scriptsize Nicolas Burrus, Alexandre Duret-Lutz, Thierry G\'eraud,
    David Lesage, and Rapha\"el Poss.  \textit{In the Proceedings of
      the Workshop on Multiple Paradigm with Object-Oriented Languages
      (MPOOL), 2003.}  }
%
  
  \medskip

%
  {\scriptsize
    \color{blue}{
      \url{http://www.lrde.epita.fr/people/theo/papers/geraud.03.mpool.pdf}
    }
  }
%

\end{frame}





%------------------------------------------------------------
\subsection{Some Limitations of OO Designs}



%........................................................................
\begin{frame}
  \frametitle{Generic Type Functions}


\begin{block}{}
  We know how to write static hierarchies with virtual types
\end{block}

% 
\bigskip
% 

Yet...

% 
\bigskip
%

\begin{block}{new objective:}
  we want to define some types that behave like generic functions
\end{block}

% 
\bigskip
\bigskip
%

For instance, something like the {\scriptsize{\textbf{boost::reverse\_iterator$<$OtherIterator$>$}}} \\
{\scriptsize \color{blue}{ \url{http://www.boost.org/libs/iterator/doc/reverse_iterator.html} } }



\end{frame}



%........................................................................
\begin{frame}
  \frametitle{Example (1/3)}


For instance, with \bfI being any iterator type, we want a
generic iterator class, \bfreverseI:

%
\bigskip
%

\begin{itemize}
\item to be defined once and for all
  \smallskip
  %
\item which inverts the behavior of \bfnext and/or \bfprev
  \smallskip
  %
\item \textit{and} which keeps the other features of \bfI
\end{itemize}

%
\bigskip
%

Note that \textbf{reverse} is just like a function of \bfI

\end{frame}



%........................................................................
\begin{frame}[fragile]
  \frametitle{Example (2/3)}

This class:
\smallskip
%

\begin{itemize}
\item should provide \bfnext iff \bfI provides \bfprev\\
  ~~~ {\scriptsize $\leadsto$ ~ we do not want any pollution by ``ghost'' methods }
\end{itemize}

%
\bigskip
%

so the snippet below is \textit{not} acceptable:

\smallskip

\begin{multicols}{2}
%
\begin{lstlisting}[basicstyle={\tiny\sffamily}]
template <typename I>
class reverse ...
{
public:
  void prev() { iter.next(); }
  next next() { iter.prev(); }
  ...
protected;
  I iter;
};
\end{lstlisting}
%
\columnbreak
%
because this class interface offers some features that might be
undesirable
%
\end{multicols}

%
\bigskip
%

{\tiny
  consequently the client gets inappropriate error messages (try for instance
  boost::reverse\_iterator with \_\_gnu\_cxx::slist and operator--). 
}

\end{frame}




%........................................................................
\begin{frame}
  \frametitle{Example (3/3)}


This class:
\smallskip
%

\begin{itemize}
\item should also provide every upcoming methods\\
  ~~~ {\scriptsize $\leadsto$ ~ i.e., methods that will be added in
    some new concepts that \bfI will model }
\end{itemize}

%
\bigskip
%

FIXME: illustration...

\end{frame}



%........................................................................
\begin{frame}
  \frametitle{Some More Examples}

Then imagine:
\smallskip
%

\begin{itemize}
\item \textbf{say\_hello$<$I$>$} that says hello when \textbf{init} is called
\smallskip
%
\item \textbf{skippy$<$I$>$} that skips $n$ elements when iterating...
\smallskip
%
\item \textbf{log$<$I, L$>$} that logs every method calls (with a
  given log policy of type \textbf{L})
\smallskip
%
\item \textbf{only\_if$<$I, P$>$} that behave like its container only has
  elements which satisfies a given predicate (of type \textbf{P}) \smallskip
%
%
\item \textbf{paired$<$I$>$} that FIXME: explain
%
\item FIXME: ...
%
\end{itemize}

\end{frame}



%........................................................................
\begin{frame}
  \frametitle{Intrinsically an OO Problem (1/3)}


\hspace*{-5mm}
\begin{minipage}{0.72\textwidth}
  \includegraphics[scale=.55]{simpledeco}
\end{minipage}
\begin{minipage}{0.28\textwidth}
  That works fine here!
  \begin{itemize}
    {\scriptsize
  \item a \textit{single} actual abstraction (Iterator)
  \item the decorator class factors code for the identity behavior
  \item the hello\_iterator class is almost empty
  \item it just does its job
  }
  \end{itemize}
\end{minipage}

\end{frame}




%........................................................................
\begin{frame}
  \frametitle{Intrinsically an OO Problem (2/3)}

\hspace*{-7mm}
\includegraphics[scale=.55]{harddeco}

\smallskip

{\scriptsize
Without duplicating code, how to
  \begin{itemize}
  \item make the \bfprev and/or \bfnext methods appear in \bfhelloiterator \\
    when they seem required?
  \item and plug this class to the relevant abstraction,\\
    either \bfBackwardIterator or \bfForwardIterator?
\end{itemize}
}

\end{frame}



%........................................................................
\begin{frame}
  \frametitle{Intrinsically an OO Problem (3/3)}

We want to express that:
\smallskip
%

\begin{itemize}
\item \bfhelloiterator transforms any \bfIterator class
\smallskip
%

\item the behavior of \bfhelloiterator mainly is the identity
\smallskip
%

\item the only method modified w.r.t. the ones of \bfI is \textbf{init}
\smallskip
%

\item \bfhelloiteratorI derives from the same
  abstractions then \bfI \smallskip
%
\end{itemize}

%
\bigskip
%

that does \textit{not} sound like something easily feasible with non-dynamic OO languages...

\end{frame}



%%% Local Variables:
%%% mode: latex
%%% eval: (ispell-change-dictionary "american")
%%% TeX-master: "slides"
%%% End:



%============================================================
\section{A Sequel to \scoop}



%------------------------------------------------------------
\subsection{Thoughts about Concepts}



%........................................................................
\begin{frame}
  \frametitle{A Concept-Oriented Design}

\vspace*{-2mm}
Consider this C++0x design:

\medskip

\includegraphics[scale=.45]{harder0x}

\end{frame}



%........................................................................
\begin{frame}
  \frametitle{Ruminations (1/3)}

It is great because:

\smallskip

  \begin{itemize}
  \item we do not have to draw explicit relations between models and concepts\\
    ~~~ {\scriptsize $\leadsto$ ~ we define \bfarrayiteratorT as a stand-alone class }\\
    ~~~ {\scriptsize $\leadsto$ ~ we maintain a very low coupling between program entities }
    % 
    \smallskip
    % 
  \item some concepts are orthogonal\\
    ~~~ {\scriptsize $\leadsto$ ~ discriminants ``browsing'' and ``accessibility'' are unrelated }
  \end{itemize}

\end{frame}




%........................................................................
\begin{frame}
  \frametitle{Ruminations (2/3)}

Yet:

\smallskip

  \begin{itemize}
  \item we are \textit{clearly} aware of \bfarrayiteratorT being an \textbf{Iterator} \\
    ~~~ {\scriptsize $\leadsto$ ~ many classes are \textit{explicitly} designed to model some domain concepts }\\
    % 
    \smallskip
    % 
  \item this design is not so far from a ``classical'' OO design based on multiple inheritance \\
    ~~~ {\scriptsize $\leadsto$ ~ and we know how to turn hierarchies into the static world }
  \end{itemize}

\smallskip

\begin{block}{Postulate}
  Static hierarchies may be a way to get in 2003 ISO/ANSI C++ such a ``C++0x''-like design.
\end{block}

\end{frame}




%........................................................................
\begin{frame}
  \frametitle{Ruminations (3/3)}

Furthermore:
    % 
    \smallskip
    % 
  \begin{itemize}
  \item when providing implementation classes (models) in C++0x, \\
    we may also want to factor some code through inheritance \\
    ~~~ {\scriptsize $\leadsto$ ~ so static hierarchies can be useful to implement models }\\
    ~~~ {\scriptsize $\leadsto$ ~ and having virtual types (resolved at compile-time) can help }
    % 
    \smallskip
    %
  \item it is well-known that structural conformance is sometimes not so great \\
    ~~~ {\scriptsize $\leadsto$ ~ how can a concept express that addition is commutative? }\\
    ~~~ {\scriptsize $\leadsto$ ~ does a 3D point model the concept of 2D point? }
  \end{itemize}

\end{frame}



%------------------------------------------------------------
\subsection{Towards a Solution}


%........................................................................
\begin{frame}
  \frametitle{The Way We Think (1/3)}

We want to implement \bfarrayiteratorT in classical OO:

\bigskip

\begin{enumerate}
{\scriptsize{

\item[] \textit{first step}
\smallskip
\item we know that \bfarrayiteratorT goes forward
\item the \bfForwardIterator {\color{red}{ abstract class}} is for iterators going forward
\item $\Rightarrow ~ $ we {\color{mediumgreen}explicitly} state that \bfarrayiteratorT derives from \bfForwardIterator

\medskip
\item[] \textit{second step}
\smallskip

\item \bfForwardIterator {\color{red}{ declares}} \bfnext
\item $\Rightarrow ~ $ we {\color{red}{define}} \bfnext in \bfarrayiteratorT

}}
\end{enumerate}

\bigskip

OK...

\end{frame}



%........................................................................
\begin{frame}
  \frametitle{The Way We Think (2/3)}

We want to implement \bfarrayiteratorT in C++-0x:

\bigskip

\begin{enumerate}
{\scriptsize{

\item[] \textit{first step}
\smallskip
\item we know that \bfarrayiteratorT goes forward
\item the \bfForwardIterator {\color{red}{ concept}} is for iterators going forward
\item $\Rightarrow ~ $ \bfarrayiteratorT shall {\color{mediumgreen}{implicitly}} model \bfForwardIterator

\medskip
\item[] \textit{second step}
\smallskip

\item \bfForwardIterator {\color{red}{ requires}} \bfnext
\item $\Rightarrow ~ $ we {\color{red}{implement}} \bfnext in \bfarrayiteratorT

}}
\end{enumerate}

\bigskip

finally that is not so different from classical OO!

\end{frame}



%........................................................................
\begin{frame}
  \frametitle{The Way We Think (3/3)}

now...

\medskip

what can we say about \bfhelloiteratorI?

\medskip

and about \bfreverseI?

\bigskip\medskip

let us try:

\medskip

\begin{enumerate}
{\scriptsize{

\item[] \textit{first step}
\smallskip
\item we know that \bfhelloiteratorI goes the same way of \bfI
\item and then we are stuck...
}}
\end{enumerate}

\bigskip\medskip

so we have to think different!

\end{frame}



%........................................................................
\begin{frame}
  \frametitle{Think Different (1/3)}

Let us rewrite:
\medskip
%
\begin{itemize}
\scriptsize
\item we know that \bfarrayiteratorT goes forward
\item $\Rightarrow ~ $ we {\color{mediumgreen}explicitly} state that \bfarrayiteratorT derives from \bfForwardIterator
\item $\Rightarrow ~ $ we {\color{red}{define}} \bfnext in \bfarrayiteratorT
\end{itemize}

%
\bigskip
%

that way:
\medskip
%

\begin{itemize}
\scriptsize
\item we {\color{mediumgreen}explicitly} state that \bfarrayiteratorT
  \begin{itemize} \scriptsize
  \item is an \bfIterator
  \item goes forward
  \end{itemize}
\item $\leadsto ~ $ \bfarrayiteratorT is {\color{mediumgreen}implicitly} a \bfForwardIterator
\item $\Rightarrow ~ $ we {\color{red}{define}} \bfnext in \bfarrayiteratorT
\end{itemize}

\end{frame}




%........................................................................
\begin{frame}
  \frametitle{Think Different (2/3)}

For \bfhelloiteratorI:
\medskip
%
\begin{itemize}
\scriptsize
\item we {\color{mediumgreen}explicitly} state that \bfhelloiteratorI
  \begin{itemize} \scriptsize
  \item is an \bfIterator
  \item goes the same way than \bfI
  \end{itemize}
\item $\leadsto ~ $ \bfhelloiteratorI is {\color{mediumgreen}implicitly} a \bfForwardIterator iff \bfI is a \bfForwardIterator
\item $\leadsto ~ $ \bfnext is {\color{mediumgreen}implicitly} defined in \bfarrayiteratorT in that case
\end{itemize}

%
\bigskip
%

more precisely:
\medskip
%
\begin{itemize}
\scriptsize
\item we {\color{mediumgreen}explicitly} state that \bfhelloiteratorI
  \begin{itemize} \scriptsize
  \item is an \bfIterator
  \item behaves like \bfI
  \end{itemize}
\item $\leadsto ~ $ we {\color{mediumgreen}implicitly} get from \bfI all we expect for \bfhelloiteratorI
\item $\Rightarrow ~ $ then we just have to {\color{red}{override}} \bfinit.
\end{itemize}

\end{frame}



%........................................................................
\begin{frame}
  \frametitle{Think Different (3/3)}

For \bfreverseI:
\medskip
%
\begin{itemize}
\scriptsize
\item we {\color{mediumgreen}explicitly} state that \bfreverseI
  \begin{itemize} \scriptsize
  \item behaves most of the time like \bfI
  \item goes forward (resp. backward) iff \bfI goes backward (resp. forward)
  \end{itemize}
\item $\Rightarrow ~ $ then we just have to {\color{red}{implement}} \bfprev and \bfnext in a  {\color{mediumgreen}non-intrusive} way.
\end{itemize}

%
\bigskip
%

Now just realize that:
\medskip
%
\begin{itemize}
\scriptsize
\item \bfreverseI can even work when \bfI is \textit{not} an iterator
\item with \textit{non-intrusive} extensions this class can handle some different ``reverse something'' cases
\item the definition of the \bfreverseI class contains about  \textit{nothing}!
\end{itemize}

\end{frame}



%........................................................................
\begin{frame}
  \frametitle{Temporary Conclusion About OOP}

\begin{block}{trouble with OOP}
  some types cannot explicitly derive from \bfForwardIterator
\end{block}

\bigskip
\bigskip

indeed, some types cannot be explicitly plugged to abstract classes
because it depends on some other information

\bigskip

{\scriptsize for instance \bfreverseI might be an iterator... or not!
  and if it is, it might be forward... or not! }


\end{frame}



%........................................................................
\begin{frame}
  \frametitle{Temporary Conclusion About Concepts}


\begin{block}{trouble with concepts}
  we cannot state that a type is a \bfForwardIterator because it
  offers the method \bfnext
\end{block}

\bigskip
\bigskip

indeed, some types are defined without knowing what methods they
actually shall feature

\bigskip

{\scriptsize for instance \bfreverseI might implement \bfnext... and
  sometimes it might not! }


\end{frame}





%........................................................................
\begin{frame}
  \frametitle{Temporary Conclusion}


We really should think different:

\bigskip

\begin{block}{Rationale}
  if a type has the \textsc{forward} \textit{{\color{red}{property}}}
  \begin{itemize}
  \item it offers the method \bfnext because
      \begin{itemize}
        \item either it implements this method
        \item or an implementation of this method is automatically fetched
      \end{itemize}
  \item and thus it is a \bfForwardIterator
  \end{itemize} 
\end{block}

\end{frame}



%........................................................................
\begin{frame}
  \frametitle{Discarding A False Solution}

Consider a generic function type \bff (such as \bfreverse), \\ its behavior
is roughly the identity of the type on which it applies.

\bigskip

Yet forming ``\bffT derives from \bfT'' is clearly \textit{not} a
solution \\ since some functions \bff are far from the identity.

% FIXME: Instead, we need a delegation mechanism (at compile-time).

\end{frame}



%........................................................................
\begin{frame}
  \frametitle{Using Delegation}

  Having \bffT means that an object with this type is constructed over
  an object with type \bfT.

  \bigskip

  {\scriptsize for instance an object with type
    \bfreversearrayiteratorT holds an object with type
    \bfarrayiteratorT }

  \bigskip

  We expect that the behavior of most methods in \bffT rely on methods
  of \bfT; so we need \textit{delegation}.

\end{frame}



%........................................................................
\begin{frame}
  \frametitle{Providing Properties}

  We sometimes said that a type has some properties (for instance the
  \textsc{forward} property).

\bigskip

\begin{block}{Property}
  A property is mapped into an associated type\\
  \textbf{+} \\
  is turned into a virtual type in order to take benefice from
  inheritance.
\end{block}

\end{frame}



%........................................................................
\begin{frame}
  \frametitle{Providing Delegation}

\begin{block}{Delegatee}
  The possible presence of a delegatee is handled as a special
  property of a type.
\end{block}

\bigskip

For instance \bfI is the delegatee of \bfreverseI, meaning that the
latter delegates some methods to the former.

\bigskip

In addition the definitions of virtual types can also be delegated.

\end{frame}




%........................................................................
\begin{frame}
  \frametitle{Providing a Category}

  We sometimes said that a type belongs to a given category of objects
  (for instance \bfIterator) without being specific about this type
  belonging to a more precise sub-category (for instance
  \bfForwardIterator).

  \bigskip

  \begin{block}{Category}
    A category is a special property of a type that identifies its
    belonging to a very large family of types.
  \end{block}

\end{frame}



%........................................................................
\begin{frame}
  \frametitle{Providing a Default Behavior}

  We sometimes said that function types have a given behavior (for
  instance \bfhelloiteratorI almost behaves like the identity of
  \bfI).

  \bigskip

\begin{block}{Behavior}
  The behavior of a type is a special property that characterizes most
  of the implementation of this type.
\end{block}

\end{frame}




%........................................................................
\begin{frame}
  \frametitle{Decoupling Implementation Classes From Abstractions}

  We sometimes cannot plug an implementation class to some abstract
  class (for instance \bfreverseI).

  \bigskip

\begin{block}{Top}
  Any implementation class derives from a class named \bftop,
  either directly or through its super classes (when we have an
  implementation hierarchy).
\end{block}

  \bigskip

  Remember that we are in a static context so, more precisely, this
  class is \textbf{top$<$Exact$>$}.

\end{frame}




%........................................................................
\begin{frame}
  \frametitle{Fetching Implementation Automatically}

  We sometimes want an implementation class to have some methods that
  cannot be directly implemented in that class (for instance, if \bfI
  is an iterator, \bfreverseI shall automatically get an \bfisvalid
  method).

  \bigskip

\begin{block}{Automatic Implementation}
  Some implementation is automatically fetched through inheritance thanks
  to a mechanism located above \bftop.
\end{block}

\end{frame}



%........................................................................
\begin{frame}
  \frametitle{Setting A Particular Implementation}

  We sometimes want an implementation class to define a very
  particular method but in a non-intrusive way (for instance, if
  \bfreverseI is a forward iterator, \bfnext shall be defined, calling
  \bfprev on the delegatee).

  \bigskip

\begin{block}{Non-intrusive Implementation}
  The mechanism located above \bftop allows for setting some
  particular implementation.
\end{block}

\end{frame}



%------------------------------------------------------------
\subsection{Abstractions and Classes in \scoop~2}


%........................................................................
\begin{frame}
  \frametitle{Abstractions}

To implement abstractions:

\medskip

\begin{itemize}
\item we stick with the 2003 ISO/ANSI C++ Standard\\
  ~ {\scriptsize{$\leadsto$ ~ so we can wait until C++-0x}}
%
  \smallskip
\item a concept is turned into an abstract class\\
  ~ {\scriptsize{$\leadsto$ ~ so it is really materialized into code (a C++ type with a name)}}
%
  \smallskip
\item a concept is parameterized by \textbf{Exact}\\
  ~ {\scriptsize{$\leadsto$ ~ so exact types are known at compile-time (static world)}}
%
  \smallskip
\item a concept refinement is handle by inheritance\\ 
  ~ {\scriptsize{$\leadsto$ ~ so we effectively get the ``is-a'' relationship}}
\end{itemize}


\end{frame}



%........................................................................
\begin{frame}
  \frametitle{Sample of Concept-Oriented Design}

\vspace*{-2mm}
Consider again this C++0x design:

\medskip

\hspace*{-5mm}
\includegraphics[scale=.45]{harder0x}

\end{frame}



%........................................................................
\begin{frame}[fragile]
  \frametitle{Code of Abstractions (1/2)}

\begin{lstlisting}[escapechar=@,basicstyle={\tiny\sffamily}]
namespace abc
{

  template <typename Exact>
  struct @\color{blue}{Iterator}@ @  :  \color{blue}{any}@<Exact>, @\color{mediumgreen}{automatic::get\_impl}@<Iterator, Exact>
  {
    @\color{mediumblue}{stc\_typename@(value);
    void init()@~~~~~~~~~~~~~~~~@{ return this->exact().impl_init(); }
    bool is_valid() const@~~~@{ return this->exact().impl_is_valid(); }
  };

  template <typename Exact>
  struct @\color{blue}{Backward\_Iterator}@ @  :  @virtual @\color{blue}{Iterator}@<Exact>, @\color{mediumgreen}{automatic::get\_impl}@<Backward_Iterator, Exact>
  {
    void prev()@~~~@{ this->exact().impl_prev();  }
  };

  template <typename Exact>
  struct @\color{blue}{Forward\_Iterator}@ @  :  @virtual @\color{blue}{Iterator}@<Exact>, @\color{mediumgreen}{automatic::get\_impl}@<Forward_Iterator, Exact>
  {
    void next()@~~~@{ this->exact().impl_next();  }
  };

  template <typename Exact>
  struct @\color{blue}{Bidirectional\_Iterator}@ @  :  @@\color{blue}{Backward\_Iterator}@<Exact>, @\color{blue}{Forward\_Iterator}@<Exact>
  {
  };

  ...
\end{lstlisting}

\end{frame}



%........................................................................
\begin{frame}
  \frametitle{Automatically Getting Implementation}

\begin{multicols}{2}
%
\vspace*{-9mm}\hspace*{-5mm}
\includegraphics[scale=.45]{corn}
%
\columnbreak
%
\begin{itemize}
\small
\item we have an ``ear of corn'' design
\item FIXME
\end{itemize}
%
\end{multicols}

\end{frame}



%........................................................................
\begin{frame}[fragile]
  \frametitle{Code of Abstractions (2/2)}

\begin{lstlisting}[escapechar=@,basicstyle={\tiny\sffamily}]
  template <typename Exact>
  struct @\color{blue}{Write\_Iterator}@ @  :  @virtual @\color{blue}{Iterator}@<Exact>, @\color{mediumgreen}{automatic::get\_impl}@<Write_Iterator, Exact>
  {
    @\color{mediumblue}{stc\_using\_from}@(Iterator, value);
    void set_value(const value& v)@~~~@{ return this->exact().impl_set_value(v); }
  };

  template <typename Exact>
  struct @\color{blue}{Write\_Only\_Iterator}@ @  :  \color{blue}{Write\_Iterator}@<Exact>
  {
  };

  template <typename Exact>
  struct @\color{blue}{Read\_Iterator}@ @  :  @virtual @\color{blue}{Iterator}@<Exact>, @\color{mediumgreen}{automatic::get\_impl}@<Read_Iterator, Exact>
  {
    @\color{mediumblue}{stc\_using\_from}@(Iterator, value);
    const value& get_value() const@~~~@{ return this->exact().impl_get_value(); }
  };

  template <typename Exact>
  struct @\color{blue}{Read\_Write\_Iterator}@ @  :  \color{blue}{Read\_Iterator}@<Exact>, @\color{blue}{Write\_Iterator}@<Exact>
  {
  };

} // end of namespace abc
\end{lstlisting}

\end{frame}





%........................................................................
\begin{frame}[fragile]
  \frametitle{Set an Implementation for a Default Behavior}


\begin{lstlisting}[escapechar=@,basicstyle={\tiny\sffamily}]
namespace abc
{
  namespace @\color{mediumgreen}{automatic}@
  {
    template <typename Exact>
    struct @\color{mediumgreen}{set\_impl}@< @\color{blue}{Iterator}@, @\color{mediumpink}{behavior::identity}@, Exact >@~~:~~@virtual any<Exact>
    {
      void impl_init()@~~~~~~~~~~~~~~~~@{ this->exact().delegatee_.init();  }
      bool impl_is_valid() const@~~~@{ return this->exact().delegatee_.is_valid(); }
    };

    template <typename Exact>
    struct @\color{mediumgreen}{set\_impl}@< @\color{blue}{Read\_Iterator}@, @\color{mediumpink}{behavior::identity}@, Exact >@~~:~~@virtual any<Exact>
    {
      @\color{mediumblue}{stc\_typename}@(value);
      const value& impl_get_value() const@~~~@{ return this->exact().delegatee_.get_value(); }
    };

    ...

  } // end of namespace abc::automatic

} // end of namespace abc
\end{lstlisting}


\end{frame}



%........................................................................
\begin{frame}
  \frametitle{A Classical Implementation Class}

We have:
\smallskip

\begin{itemize}
\scriptsize
\item we {\color{mediumgreen}{explicitly}} state that \bfarrayiteratorT
  \begin{itemize} \scriptsize
  \item is an \bfIterator
  \item goes forward
  \item allows to assign a value (write in the container)
  \end{itemize}
\item $\Rightarrow ~ $ we have to implement the corresponding methods.
\end{itemize}

\bigskip

So:
\smallskip

\begin{itemize}
  \scriptsize
\item this class derives from \bftop 
\item we do \textit{not} have to draw an explicit link towards
  \bfForwardIterator or to \bfWriteOnlyIterator
\item we can omit setting the \textsc{read} and \textsc{backward} to
  'false'\\
  ~~~ {\scriptsize $\leadsto$ ~ a property is verified when its value is set to ``true'' }\\
  ~~~ {\scriptsize $\leadsto$ ~ so if a property is unset, its value is not ``true'' }\\
\end{itemize}

\end{frame}



%........................................................................
\begin{frame}[fragile]
  \frametitle{Code of \bfarrayiteratorT}

\begin{lstlisting}[escapechar=@,basicstyle={\tiny\sffamily}]
# define templ@~~~~~~~~@template <typename T>
# define classname@~~@array_iterator
# define current@~~~~~~\color{red}{array\_iterator$<$T$>$}@
# define super@~~~~~~~~\color{mediumgreen}{top}@< current >

@\color{mediumblue}{stc\_Header}@;
  typedef stc::is<@\color{blue}{Iterator}@> category;
  typedef T value;
  typedef stc::true_ @\color{blue}{forward}@;
  typedef stc::true_ @\color{blue}{write}@;
@\color{mediumblue}{stc\_End}@;

template <typename T>
class array_iterator : public super
{
public:
  @\color{mediumblue}{stc\_using}@(value);
  void impl_init()@~~~~~~~~~~~~~~~~~~~~~~~~~~~~~@{ i_ = 0; }
  void impl_next()@~~~~~~~~~~~~~~~~~~~~~~~~~~~~@{ i_ = i_ + 1; }
  bool impl_is_valid() const@~~~~~~~~~~~~~~~~@{ return i_ >= 0 and i_ < n_; }
  void impl_set_value(const value& v)@~~~@{ this->v_ = v; }

  array_iterator(int n)@~~~~~~~~~~~~~~~~~~~~~~~~@: n_(n) {}
  int index() const@~~~~~~~~~~~~~~~~~~~~~~~~~~~~@{ return i_; }
protected:
  int i_, n_;
};
\end{lstlisting}

\end{frame}




%........................................................................
\begin{frame}
  \frametitle{A Generic Function}

Remember:
\smallskip

\begin{itemize}
\scriptsize
\item we {\color{mediumgreen}explicitly} state that \bfhelloiteratorI
  \begin{itemize} \scriptsize
  \item is an \bfIterator
  \item behaves by default like \bfI
  \end{itemize}
\item $\leadsto ~ $ we {\color{mediumgreen}implicitly} get from \bfI all we expect for \bfhelloiteratorI
\item $\Rightarrow ~ $ then we just have to {\color{red}{override}} \bfinit.
\end{itemize}

\end{frame}



%........................................................................
\begin{frame}[fragile]
  \frametitle{Code of \bfhelloiteratorI}

\begin{lstlisting}[escapechar=@,basicstyle={\tiny\sffamily}]
# define templ@~~~~~~~~@template <typename I>
# define classname@~~@hello_iterator
# define current@~~~~~~\color{red}{hello\_iterator$<$I$>$}@
# define super@~~~~~~~~\color{mediumgreen}{top}@< current >

@\color{mediumblue}{stc\_Header}@;
  typedef stc::is<@\color{blue}{Iterator}@> category;
  typedef @\color{mediumpink}{tag::identity}@ behavior;
  typedef I delegatee;
@\color{mediumblue}{stc\_End}@;

template <typename I>
class hello_iterator : public super
{
public:
  I& delegatee_;
  hello_iterator(I& iter) : delegatee_(iter) {}  // ctor

  // overriding
  void impl_init()
  {
    std::cout << "hello!" << std::endl;
    super::impl_init();
  }
};
\end{lstlisting}

\end{frame}





%........................................................................
\begin{frame}[fragile]
  \frametitle{Another Generic Function}

Remember:
\smallskip

\begin{itemize}
\scriptsize
\item we {\color{mediumgreen}explicitly} state that \bfreverseT
  \begin{itemize} \scriptsize
  \item behaves most of the time like \bfT
  \item goes forward (resp. backward) iff \bfI goes backward (resp. forward)
  \end{itemize}
\item $\Rightarrow ~ $ then we just have to {\color{red}{implement}} \bfprev and \bfnext in a  {\color{mediumgreen}non-intrusive} way.
\end{itemize}

\end{frame}




%........................................................................
\begin{frame}[fragile]
  \frametitle{Code of \bfreverseT (1/2)}

\begin{multicols}{2}
%
\begin{lstlisting}[escapechar=@,basicstyle={\tiny\sffamily}]
// first, name the behavior

namespace behavior { struct reverse; }


// then, define the class

# define templ@~~~~~~~~@template <typename T>
# define classname@~~@reverse
# define current@~~~~~~\color{red}{reverse$<$T$>$}@
# define super@~~~~~~~~\color{mediumgreen}{top}@< current >

@\color{mediumblue}{stc\_Header}@;
  typedef T delegatee;
  typedef @\color{mediumpink}{behavior::reverse}@ behavior;
@\color{mediumblue}{stc\_End}@;

template <typename T>
class reverse : public super
{
public:
  T& delegatee_; // mandatory for delegation
  reverse(T& d) : delegatee_(d) {}
};
\end{lstlisting}
%
\columnbreak
%
{\scriptsize{ We can extend (non-intrusively) this class type with
    new virtual types: }} \smallskip

\begin{lstlisting}[escapechar=@,basicstyle={\tiny\sffamily}]
  @\color{mediumblue}{stc\_Header\_Extension}@(forward);
    typedef stc_prop(T, backward) ret;
  @\color{mediumblue}{stc\_End}@;

  @\color{mediumblue}{stc\_Header\_Extension}@(backward);
    typedef stc_prop(T, forward) ret;
  @\color{mediumblue}{stc\_End}@;
\end{lstlisting}

\smallskip {\scriptsize{ meaning that, \bfreverseT is forward iff \bfT
    has the backward property }}
%
\end{multicols}

\end{frame}



%........................................................................
\begin{frame}[fragile]
  \frametitle{Code of \bfreverseT (2/2)}

\begin{lstlisting}[escapechar=@,basicstyle={\tiny\sffamily}]
namespace abc
{

  namespace @\color{mediumgreen}{automatic}@
  {
    // default behavior is 'identity'

    template <template <class> class @\textit{abstraction}@, typename Exact>
    struct @\color{mediumgreen}{set\_impl}@< @\color{blue}{\textit{abstraction}}@, @\color{mediumpink}{behavior::reverse}@, Exact >@~~:~~\color{mediumgreen}{impl}@< @\color{blue}{\textit{abstraction}}@, @\color{mediumpink}{tag::identity}@, Exact >
    {};

    // particular implementations

    template <typename Exact>
    struct @\color{mediumgreen}{set\_impl}@< @\color{blue}{Backward\_Iterator}@, @\color{mediumpink}{behavior::reverse}@, Exact > : virtual any<Exact>
    {
      void impl_prev()@~~~@{ this->exact().delegatee_.next();  }@~~~~@// 'prev' means 'next' 
    };

    template <typename Exact>
    struct @\color{mediumgreen}{set\_impl}@< @\color{blue}{Forward\_Iterator}@, @\color{mediumpink}{behavior::reverse}@, Exact > : virtual any<Exact>
    {
      void impl_next()@~~~@{ this->exact().delegatee_.prev();  }@~~~~@// 'next' means 'prev'
    };

  } // end of namespace abc::automatic

} // end of namespace abc
\end{lstlisting}

\end{frame}





%........................................................................
\begin{frame}[fragile]
  \frametitle{Sample use}

\begin{multicols}{2}
%
\begin{lstlisting}[escapechar=@,basicstyle={\tiny\sffamily}]
template <typename @\greenI@>
void echo(const @\color{blue}{abc::Forward\_Iterator}@<@\greenI@>&)
{
  std::cout << "@\textbf{Forward Iterator}@" << std::endl;
}

template <typename @\greenI@>
void echo(const @\color{blue}{abc::Backward\_Iterator}@<@\greenI@>&)
{
  std::cout << "@\textbf{Backward Iterator}@" << std::endl;
}
\end{lstlisting}
%
\columnbreak
%
\begin{lstlisting}[escapechar=@,basicstyle={\tiny\sffamily}]
int main()
{
  typedef abc::array_iterator<int> iterator_t;
  
  iterator_t i(8);
  echo(i);@~~~@// {\color{darkgray}{gives:}} \textbf{Forward Iterator}

  abc::reverse<iterator_t> j(i);
  echo(j);@~~~@// {\color{darkgray}{gives:}} \textbf{Backward Iterator}
  
  i.init();
  std::cout << i.index() << std::endl;
  // {\color{darkgray}{gives: 0}}
  
  j.prev();@~~~@// {\color{darkyellow}{actually means ``i.next()''}}
  std::cout << i.index() << std::endl;
  // {\color{darkgray}{gives: 1}}
}
\end{lstlisting}
%
\end{multicols}

\end{frame}



%........................................................................
\begin{frame}
  \frametitle{Between \bftop and Abstractions (1/2)}


\begin{itemize}
\item an implementation class derives from \bftop\\
  ~~~ {\tiny $\leadsto$ ~ so, unlike in classical OO, we do \textit{not} plug the class into some abstract class(es)}
  \medskip
  % 
\item this class might be related to some abstractions\\
  ~~~ {\tiny $\leadsto$ ~ and these possible relationships looks \textit{implicit} to the client}
  \medskip
  % 
\item that depends upon:
    \smallskip
    % 
  \begin{itemize}
    \scriptsize
  \item the value of the class category\\
    ~~~ {\tiny $\leadsto$ ~ for instance the category of \bfarrayiteratorT is set to \textbf{stc::is$<$Iterator$>$}}
    \smallskip
    % 
  \item \textit{and} some computation over the class virtual types\\
    ~~~ {\tiny $\leadsto$ ~ for instance to be a \bfForwardIterator the \textsc{forward} property shall be true}
    \smallskip
    % 
  \item \textit{and possibly} the nature of the delegatee\\
    ~~~ {\tiny $\leadsto$ ~ if it exists and provides relevant information}\\
    ~~~ {\tiny $\leadsto$ ~ for instance \bfT for the class \bfreverseT}
  \end{itemize}
  %
\end{itemize}

\end{frame}



%........................................................................
\begin{frame}
  \frametitle{Between \bftop and Abstractions (1/2)}

Above \bftop
\medskip

\begin{itemize}
  \scriptsize
\item an equipment exists to link implementation classes with abstractions\\
    ~~~ {\tiny $\leadsto$ ~ this equipment is a meta-program imported by the macro {\color{mediumblue}{stc\_equip\_namespace}}}
  \smallskip
  % 
\item (and that is also the means to automatically fetch some default implementations)\\
  ~~~ {\tiny $\leadsto$ ~ so we can have generic function types like \bfreverseT}
  \smallskip
  % 
\item yet the designer has to explain how properties and abstractions are related\\
  ~~~ {\tiny $\leadsto$ ~ for instance the \textsc{forward} property goes with the \bfForwardIterator abstraction }
\end{itemize}

\medskip

  \begin{block}{Plugging rules}
    The relations between properties and abstractions are implemented
    in a declarative way.
  \end{block}

\end{frame}



%........................................................................
\begin{frame}
  \frametitle{The Running Concept-Oriented Design}

{\scriptsize This design feature two \textit{separate} concept hierarchies:}

\medskip

\includegraphics[scale=.4]{selectors}

\medskip

{\scriptsize we then have to define two \textit{selectors}, one per hierarchy}

\end{frame}



%........................................................................
\begin{frame}[fragile]
  \frametitle{Code of Selector 1 (iterator browsing mode)}

\begin{lstlisting}[escapechar=@,basicstyle={\tiny\sffamily}]
namespace abc
{
  namespace @\color{mediumgreen}{internal}@
  {

    typedef @\color{mediumpink}{selector}@<@\textit{\color{blue}{Iterator}}@, @\color{mediumpink}{1}@> @\color{mediumpink}{Iterator\_browsing}@;
  
    template <typename Exact>
    struct @\color{darkyellow}{case\_}@< @\color{darkpink}{Iterator\_browsing}@, Exact,  @\color{darkyellow}{~1}@ > : @\color{darkyellow}{where\_}@< mlc::and_< stc_is(@\color{mediumblue}{forward}@), stc_is(@\color{mediumblue}{backward}@) > >
    {
      typedef @\color{blue}{Bidirectional\_Iterator}@<Exact> ret;
    };
  
    template <typename Exact>
    struct @\color{darkyellow}{case\_}@< @\color{darkpink}{Iterator\_browsing}@, Exact,  @\color{darkyellow}{~2}@ > : @\color{darkyellow}{where\_}@< stc_is(@\color{mediumblue}{forward}@) >
    {
      typedef @\color{blue}{Forward\_Iterator}@<Exact> ret;
    };
  
    template <typename Exact>
    struct @\color{darkyellow}{case\_}@< @\color{darkpink}{Iterator\_browsing}@, Exact,  @\color{darkyellow}{~3}@ > : @\color{darkyellow}{where\_}@< stc_is(@\color{mediumblue}{backward}@) >
    {
      typedef @\color{blue}{Backward\_Iterator}@<Exact> ret;
    };
    
  } // end of namespace abc::internal

} // end of namespace abc
\end{lstlisting}

\end{frame}




%........................................................................
\begin{frame}[fragile]
  \frametitle{Code of Selector 2 (iterator data accessibility)}

\begin{lstlisting}[escapechar=@,basicstyle={\tiny\sffamily}]
namespace abc
{
  namespace @\color{mediumgreen}{internal}@
  {

    typedef @\color{mediumpink}{selector}@<@\textit{\color{blue}{Iterator}}@, @\color{mediumpink}{2}@> @\color{mediumpink}{Iterator\_accessibility}@;
  
    template <typename Exact>
    struct @\color{darkyellow}{case\_}@< @\color{darkpink}{Iterator\_accessibility}@, Exact,  @\color{darkyellow}{~1}@ > : @\color{darkyellow}{where\_}@< mlc::and_< stc_is(@\color{mediumblue}{read}@), stc_is(@\color{mediumblue}{write}@) > >
    {
      typedef @\color{blue}{Read\_Write\_Iterator}@<Exact> ret;
    };
  
    template <typename Exact>
    struct @\color{darkyellow}{case\_}@< @\color{darkpink}{Iterator\_accessibility}@, Exact,  @\color{darkyellow}{~2}@ > : @\color{darkyellow}{where\_}@< mlc::and_< stc_is(@\color{mediumblue}{read}@), stc_is_not(@\color{mediumblue}{write}@) > >
    {
      typedef @\color{blue}{Read\_Only\_Iterator}@<Exact> ret;
    };
  
    template <typename Exact>
    struct @\color{darkyellow}{case\_}@< @\color{darkpink}{Iterator\_accessibility}@, Exact,  @\color{darkyellow}{~3}@ > : @\color{darkyellow}{where\_}@< mlc::and_< stc_is(@\color{mediumblue}{write}@), stc_is_not(@\color{mediumblue}{read}@) > >
    {
      typedef @\color{blue}{Write\_Only\_Iterator}@<Exact> ret;
    };
    
  } // end of namespace abc::internal

} // end of namespace abc
\end{lstlisting}

\end{frame}



%%% Local Variables:
%%% mode: latex
%%% eval: (ispell-change-dictionary "american")
%%% TeX-master: "slides"
%%% End:





%------------------------------------------------------------
\subsection{At a Glance}


%........................................................................
\begin{frame}
  \frametitle{Recap}

\hspace*{-8mm}
\includegraphics[scale=.51]{glance}

\smallskip

{\scriptsize
with \scoop we are able to design an OO hierarchy to provide the user with implementation classes
(that allows for factoring code, the use of statically resolved virtual types and delegation being \textit{very} convenient)
}

\end{frame}



%........................................................................
\begin{frame}
  \frametitle{Zoom In}
\begin{multicols}{2}
%
\includegraphics[scale=.51]{zoomin}
%
\columnbreak
%
\begin{itemize}
  \scriptsize
\item when creating an implementation class (C), the programmer does
  \textit{not} draw an explicit link towards
  \begin{itemize} \scriptsize
    \item abstractions ($A_i$)
    \item and default implementation classes (impl$< \!\! A_i \!\! >$)
  \end{itemize}
  \smallskip
  % 
\item the pink frame can be understood as an \textit{equipment} that allows for
  \begin{itemize} \scriptsize
  \item relating implementation classes with abstractions
  \item fetching default implementations for generic function types
  \end{itemize}
\end{itemize}
%
\end{multicols}

\end{frame}




%........................................................................
\begin{frame}
  \frametitle{\textit{A posteriori} Justification (1/3)}

  {\scriptsize{ Let us consider to move a \scoop~2 design towards a
    more ``common'' generic programming design.  We then remove the
    contents of abstract classes (and rename methods):
  }}

  \begin{center}
    \includegraphics[scale=.41]{alt}
  \end{center}

\end{frame}




%........................................................................
\begin{frame}
  \frametitle{\textit{A posteriori} Justification (2/3)}


  The intermediate hierarchy of empty classes is kept since it is:

  \begin{center}
    \textit{the} mechanism to fetch default implementations
  \end{center}

  consequently the design structure has not significantly
  changed.

  \bigskip\bigskip

  Unfortunately making these classes empty implies renaming methods
  (removing the ``impl\_'' suffix), so

  \begin{center}
    we are \textit{no more} featuring the classical OO dispatch
  \end{center}
  
  and that is a great loss.

\end{frame}



%........................................................................
\begin{frame}
  \frametitle{\textit{A posteriori} Justification (3/3)}


  The role of abstract classes in \scoop~2 is manyfold:
  \medskip
  %
  \begin{itemize}
  \item they expose their interface, i.e., (abstract) methods and types\\
    ~~~ {\scriptsize $\leadsto$ ~ and they provide code with \textit{named} abstractions }
    \smallskip
    % 
  \item they statically ``resolve'' their interface \\
    ~~~ {\scriptsize $\leadsto$ ~ thanks to {\color{mediumblue}{stc\_typename}} and {\color{darkgray}{this-$>$exact().impl\_}} }
    \smallskip
    % 
  \item they are part of the default implementation fetching mechanism \\
    ~~~ {\scriptsize $\leadsto$ ~ and we need this mechanism to have generic function types }
    \smallskip
  \end{itemize}

\end{frame}




%........................................................................
\begin{frame}
  \frametitle{Evolving Towards C++-0x}

  {\scriptsize{ the \scoop abstract classes, re-named as
      \textbf{I$_i$}, are now the ``interface classes'' (they provide
      the proper interfaces for implementation classes) }}

\vspace*{-2mm}
  \begin{center}
    \includegraphics[scale=.41]{evolution}
  \end{center}

\vspace*{-3mm}
{\scriptsize{
    the implementation class \textbf{C} is statically plugged into \textbf{I$_1$} so \textbf{C} models the concept \textbf{A$_1$}
}}

\end{frame}


%============================================================
\section*{Conclusion}

%........................................................................
\begin{frame}
  \frametitle{Conclusion}

FIXME

\end{frame}


%........................................................................
\begin{frame}
  \frametitle{References}

FIXME

\end{frame}





%######################################################''


\end{document}

%%% Local Variables:
%%% mode: latex
%%% eval: (ispell-change-dictionary "american")
%%% TeX-master: t

% LocalWords:  OOP namespace stc basicstyle mediumpink classname templ const
% LocalWords:  abc mediumblue impl escapechar darkyellow discriminants prev
% LocalWords:  vtypes Desugared
